\documentclass[12pt]{article}
\usepackage[utf8]{inputenc}
\usepackage{vmargin}
\usepackage{fancyhdr}
\usepackage{graphics}
\usepackage{../../icacit}
\usepackage[margin=2.5cm,headheight=200pt,includeheadfoot]{geometry}

\setpapersize{A4}
\setmargins{2.5cm}       % margen izquierdo
{0.5cm}                        % margen superior
{16.5cm}                      % anchura del texto
{23.42cm}                    % altura del texto
{10pt}                           % altura de los encabezados
{1cm}                           % espacio entre el texto y los encabezados
{0pt}                             % altura del pie de página
{2cm}                           % espacio entre el texto y el pie de página


\begin{document}

\sylabusHeader
\sylabusTitle

\curso
{1003230} %Código del curso
{SISTEMAS OPERATIVOS} %Nombre del curso
{2018-B} %Semestre

\creditosHoras
{4} %Número de creditos
{2} %Horas teoría
{2} %Horas práctica
{} %Horas teórico-practica
{2} %Horas laboratorio
{6} %Total de horas

\instructor
{Ana Maria Cuadros Valdivia}

\libro
{Introduction to Algorithms, (3th ed.)} %Título del libro
{Cormen, T. H., Leiserson, C. E., Rivest, R. L., and Stein, C} %Autor del libro
{2013} %Año del libro

\libroSecundario
{An Introduction to the Analysis of Algorithms} %Título del libro
{Sedgewick and Flajolet, 2013} %Autor del libro
{2013} %Año del libro

\begin{datosCurso}
    \begin{descripcion}
        El curso de Análisis y Diseño de Algoritmos tiene como propósito que el estudiante aprenda los principios fundamentales del diseño, implementación y análisis de algoritmos  para la solución de problemas computacionales.
        
    \end{descripcion}
    \begin{requisitos}
        \newItemReq{CS210}{Algoritmos y Estructura de Datos}
        
    \end{requisitos}
    \ObligatorioElectivo
    {X} %Marcar esta si es obligatorio 
    {} %Marcar esta si es electivo
\end{datosCurso}

\begin{objetivosCurso}
    \begin{resultadosEspecificos}
        Elaborar un diseño (conceptual y lógico) apropiado de una base de datos que permita almacenar información clave y estratégica para la empresa, la cual constituye un punto de partida para  construir una base de datos para un sistema de información. Implementar una base de datos relacional basados en los requerimientos de información de una organización, optimizando el acceso a la base de datos de las mismas bajo criterios de normalización.
    \end{resultadosEspecificos}
    \begin{resultadosEstudiante}
        1) La capacidad de aplicar principios de diseño y desarrollo en la construcción de sistemas de software de diversa complejidad 2) Aplica en un nivel intermedio.
    \end{resultadosEstudiante}
\end{objetivosCurso}

\begin{ListaTemas}
\nextUnidad{Identificación de requerimientos}
\nextUnidad{Diseño de la base de datos}
\nextUnidad{Construcción de bases de datos}
\end{ListaTemas}


\end{document}