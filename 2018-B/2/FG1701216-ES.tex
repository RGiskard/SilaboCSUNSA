\documentclass[a4paper,8pt]{article}
\usepackage[utf8]{inputenc}
\usepackage{tabularx}
\usepackage{setspace}
\usepackage{../../silabus}

\begin{document}

%%Setear variables
\setPeriodoAcademico{2018-B}
\setNombreAsignatura{Realidad Nacional}
\setNombreProfesor{}
\setGradoProfesorAbreviado{}
\sylabusHeader

\academicaTable
{Ciencia de la Computación} %Escuela Profesional
{FG1701216} %Código de la asignatura
{2$^{do}$ Semestre.} %Número del semestre
{Semestral} %Características
{17 Semanas} %Duración
{1 HT} %Número de horas teóricas
{2 HP} %Número de horas prácticas
{0} %Número de horas seminarios
{}  %Número de horas laboratorio
{2} %Número de horas Teórico-práctico
{2} %Número de créditos
{Ninguno} % Prerrequisitos (separados por comas)

\administrativaTable
{Doctor} %Grado académico del profesor
{Ingeniería de Sistemas e Informática} %Departamento académico
{6} %Número de horas totales
{2} %Número de horas - lunes
{-} %Número de horas - martes
{2} %Número de horas - miercoles
{2} %Número de horas - jueves
{-} %Número de horas - viernes
{101} %Aula de clase - lunes
{-} %Aula de clase - martes
{101} %Aula de clase - miercoles
{101} %Aula de clase - jueves
{-} %Aula de clase - viernes


\begin{fundamentacion}
La formación integral del alumno supone una adecuada valoración histórica de la realidad nacional de modo que su accionar profesional esté integrado y articulado con la identidad cultural peruana, que genera el compromiso de hacer de nuestra sociedad un ámbito más humano, solidario y justo.

\end{fundamentacion}

\begin{sumilla}
\item 
\item 
\item 
\item 

\end{sumilla}

\begin{competenciasAsignatura}
\item \ShowCompetence{C20}{e,n}

\end{competenciasAsignatura}

\begin{contenidos}


%inicio unidad
\nextUnidad{}
\nextCapitulo{}
\nextTema{Aspectos conceptuales relevantes para elaboración de las matrices analíticas.}
\nextTema{Cultura.}
\nextTema{Identidad.}
\nextTema{Nación.}
\nextTema{Sociedad.}
\nextTema{Estado.}
\nextTema{Normas para elaboración de matrices.}
\nextTema{El imperio de los Incas.}
\nextTema{Repaso de  aspectos socio-culturales más importantes.}
\nextTema{Elaboración de la matriz del imperio Inca.}
\nextTema{Conquista española.}
\nextTema{?`Encuentro o choque de las culturas?.}
\nextTema{Hacia una comprensión integral del fenómeno.}
\nextTema{Debate conceptual.}
\nextTema{Elaboración de matriz: cultura española.}
\nextTema{Virreinato.}
\nextTema{Repaso de  aspectos socio-culturales más importantes.}
\nextTema{Surgimiento de la identidad nacional peruana al calor de la Fe Católica.}
\nextTema{Elaboración de matriz: cultura virreinal.}

\cite{Belaunde1965}, \cite{Messori1998}, \cite{Morande1987}, \cite{Vargas1996} 


%inicio unidad
\nextUnidad{}
\nextCapitulo{}
\nextTema{El proceso de la emancipación peruana.}
\nextTema{Hacia una compresión integral del fenómeno.}
\nextTema{Debate conceptual.}
\nextTema{Elaboración de matriz.}

\cite{Pease1999}, \cite{Basadre1994}, \cite{Vargas1996} 


%inicio unidad
\nextUnidad{}
\nextCapitulo{}
\nextTema{Primeros cambios culturales.}
\nextTema{Inicio del proceso secularizador de la cultura.}
\nextTema{Primera República y Militarismo.}
\nextTema{Repaso de  aspectos socio-culturales más importantes.}
\nextTema{Elaboración de matriz.}
\nextTema{Prosperidad Falaz.}
\nextTema{Repaso de  aspectos socio-culturales más importantes.}
\nextTema{Elaboración de matriz.}
\nextTema{Guerra con Chile.}
\nextTema{Repaso de  aspectos socio-culturales más importantes.}
\nextTema{Elaboración de matriz.}

\cite{Pease1999}, \cite{Vargas1996} 


%inicio unidad
\nextUnidad{}
\nextCapitulo{}
\nextTema{Principales ideologías políticas en el siglo XX en contrapunto con los principios de la Doctrina Social de la Iglesia.}
\nextTema{Ciclo liberal.}
\nextTema{Repaso de  aspectos socio-culturales más importantes.}
\nextTema{Elaboración de matriz.}
\nextTema{Ciclo Nacional-Populista.}
\nextTema{Primer subciclo (1930-1948).}
\nextTema{Repaso de  aspectos socio-culturales más importantes.}
\nextTema{Elaboración de matriz.}
\nextTema{Segundo subciclo (1948-1968).}
\nextTema{Repaso de  aspectos socio-culturales más importantes.}
\nextTema{Elaboración de matriz.}
\nextTema{Tercer subciclo (1968-1980).}
\nextTema{Repaso de  aspectos socio-culturales más importantes.}
\nextTema{Elaboración de matriz.}
\nextTema{Cuarto subciclo (1980-1990).}
\nextTema{Repaso de  aspectos socio-culturales más importantes.}
\nextTema{Elaboración de matriz.}
\nextTema{Ciclo Neoliberal (1990- ?`?).}
\nextTema{Repaso de  aspectos socio-culturales más importantes.}
\nextTema{Elaboración de matriz.}
\nextTema{Situación de la Nación Peruana.}
\nextTema{Recuperación de la integración y la solidaridad socio-cultural.}

\cite{Pease1999}, \cite{Delapuente2006}, \cite{Quiroz2006} 





\end{contenidos}




\begin{estrategiasEnsenanza}
    \begin{metodos}
        Método expositivo en las clases teóricas \\
        Método de elaboración conjunta en los seminarios taller y en la elaboración del proyecto de investigación.
    \end{metodos}
    \begin{medios}
        Pizarra acrílica, plumones, cañón multimedia, material de laboratorio, videos, software.
    \end{medios}
    \begin{formasOrganizacion}
        %Se pone los que se necesiten
        \newItemFO{Clases Teóricas}{Desarrollo de los conceptos teóricos}
        \newItemFO{Seminarios}{Algo...}
        \newItemFO{Prácticas}{Algo...}
        \newItemFO{Laboratorio}{Aplicación de los conceptos vistos es clases teóricas.}
        \newItemFO{Otros}{Algo...}
    \end{formasOrganizacion}
    \begin{programacion}
        \newItemFO{Investigación Formativa}{Implementación de Sistema Computacional Web usando una base de datos relacional normalizada}
        \newItemFO{Responsabilidad Social}{Generar videos para la enseñanza de implementación de bases de datos y que sean disponibilizados de la población}
    \end{programacion}
    \begin{segumientoAprendizaje}
        Aquí va el seguimiento del aprendizaje
    \end{segumientoAprendizaje}
\end{estrategiasEnsenanza}


\begin{cronogramaAcademico}
    \newItemCA{Tema1}{Edward Hinojosa Cárdenas}{10}  %Tema/Evaluación - Docente - Porcentaje acumulado
    \newItemCA{Tema2}{Edward Hinojosa Cárdenas}{16}
    \newItemCA{Tema3}{Edward Hinojosa Cárdenas}{20}
    \newItemCA{Tema4}{Edward Hinojosa Cárdenas}{25}
    \newItemCA{Tema5}{Edward Hinojosa Cárdenas}{33}
    \newItemCA{Tema6}{Edward Hinojosa Cárdenas}{37}
    \newItemCA{Tema7}{Edward Hinojosa Cárdenas}{40}
    \newItemCA{Tema8}{Edward Hinojosa Cárdenas}{45}
    \newItemCA{Tema9}{Edward Hinojosa Cárdenas}{50}
    \newItemCA{Tema10}{Edward Hinojosa Cárdenas}{53}
    \newItemCA{Tema11}{Edward Hinojosa Cárdenas}{58}
    \newItemCA{Tema12}{Edward Hinojosa Cárdenas}{64}
    \newItemCA{Tema13}{Edward Hinojosa Cárdenas}{69}
    \newItemCA{Tema14}{Edward Hinojosa Cárdenas}{76}
    \newItemCA{Tema15}{Edward Hinojosa Cárdenas}{84}
    \newItemCA{Tema16}{Edward Hinojosa Cárdenas}{93}
    \newItemCA{Tema17}{Edward Hinojosa Cárdenas}{100}
\end{cronogramaAcademico}

\begin{estrategiasEvaluacion}
    \begin{evaluacionContinua}
        Práctica y Laboratorios en cada clase sobre los temas realizados, tanto para el primer parcial (EC1), segundo parcial (EC2) y tercer parcial (EC3).
    \end{evaluacionContinua}
    \begin{evaluacionPeriodica}
        \newItemEP{Primer Examen}{ponderación}
        \newItemEP{Segundo Examen}{ponderación}
        \newItemEP{Tercer Examen}{}
    \end{evaluacionPeriodica}
    \begin{cronogramaEvaluacion}
        \newItemCE{11/5/2019}{11/5/2019}{11/5/2019}{30\%}
        \newItemCE{11/8/2019}{11/8/2019}{11/8/2019}{30\%}
        \newItemCE{11/12/2019}{11/12/2019}{11/12/2019}{40\%}
    \end{cronogramaEvaluacion}
    \begin{tipoEvaluacion}
        Tipo de evaluación
    \end{tipoEvaluacion}
    \begin{instrumentosEvaluacion}
        Instrumentos de evaluación
    \end{instrumentosEvaluacion}
\end{estrategiasEvaluacion}

\begin{requisitosAprobacion}
\item El alumno tendrá derecho a observar o en su defecto a ratificar las notas consignadas en sus evaluaciones, después de ser entregadas las mismas por parte del profesor, salvo el vencimiento de plazos para culminación del semestre académico, luego del mismo, no se admitirán reclamaciones,
alumno que no se haga presente en el día establecido, perderá su derecho a reclamo.
\item Para aprobar ...
\end{requisitosAprobacion}

\bibliography{FG220.bib}
\bibliographystyle{apalike}

\fecha
\firma

\end{document}


