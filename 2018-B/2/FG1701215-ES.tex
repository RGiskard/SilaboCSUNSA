\documentclass[a4paper,8pt]{article}
\usepackage[utf8]{inputenc}
\usepackage{tabularx}
\usepackage{setspace}
\usepackage{../../silabus}

\begin{document}

%%Setear variables
\setPeriodoAcademico{2018-B}
\setNombreAsignatura{Comunicación Integral}
\setNombreProfesor{}
\setGradoProfesorAbreviado{}
\sylabusHeader

\academicaTable
{Ciencia de la Computación} %Escuela Profesional
{FG1701215} %Código de la asignatura
{2$^{do}$ Semestre.} %Número del semestre
{Semestral} %Características
{17 Semanas} %Duración
{2 HT} %Número de horas teóricas
{2 HP} %Número de horas prácticas
{0} %Número de horas seminarios
{}  %Número de horas laboratorio
{2} %Número de horas Teórico-práctico
{3} %Número de créditos
{Ninguno} % Prerrequisitos (separados por comas)

\administrativaTable
{Doctor} %Grado académico del profesor
{Ingeniería de Sistemas e Informática} %Departamento académico
{6} %Número de horas totales
{2} %Número de horas - lunes
{-} %Número de horas - martes
{2} %Número de horas - miercoles
{2} %Número de horas - jueves
{-} %Número de horas - viernes
{101} %Aula de clase - lunes
{-} %Aula de clase - martes
{101} %Aula de clase - miercoles
{101} %Aula de clase - jueves
{-} %Aula de clase - viernes


\begin{fundamentacion}
Para lograr una eficaz comunicación en el ámbito personal y profesional, es prioritario el manejo adecuado de la Lengua en forma oral y escrita. Se justifica, por lo tanto, que los alumnos  conozcan, comprendan y apliquen los aspectos conceptuales y operativos de su idioma, para el desarrollo de sus habilidades comunicativas fundamentales: Escuchar, hablar, leer y escribir.
En consecuencia el ejercicio permanente y el aporte de los fundamentos contribuyen grandemente en la formación académica y, en el futuro, en el desempeño de su profesión

\end{fundamentacion}

\begin{sumilla}
\item Presentación del Curso
\item Características de la Escritura Académica
\item Estrategias de Lectura
\item Estructura del Texto
\item Estructura de Párrafos
\item Características del párrafo
\item Esquema de síntesis
\item Texto argumentativo vs. expositivo
\item Proceso de Redacción
\item Entrega Virtual
\item Proceso de Redacción
\item Citas
\item Tipos de párrafos
\item Aproximación a características de la exposición oral
\item Párrafo Comparativo
\item Características de la exposición oral y tipos de párrafo
\item Redacción de texto completo
\item Teoría: redacción de textos completos
\item Exposiciones

\end{sumilla}

\begin{competenciasAsignatura}
\item \ShowCompetence{C17}{f,h,n}
\item \ShowCompetence{C20}{f,n}
\item \ShowCompetence{C24}{f,h}

\end{competenciasAsignatura}

\begin{contenidos}


%inicio unidad
\nextUnidad{Presentación del Curso}
\nextCapitulo{Presentación del Curso}
\nextTema{Comunicación Oral y Escrita I}

\cite{Real} 


%inicio unidad
\nextUnidad{Características de la Escritura Académica}
\nextCapitulo{Características de la Escritura Académica}
\nextTema{Características de la Escritura Académica}

\cite{Real} 


%inicio unidad
\nextUnidad{Estrategias de Lectura}
\nextCapitulo{Estrategias de Lectura}
\nextTema{Características de las estrategias de lectura.}

\cite{Real} 


%inicio unidad
\nextUnidad{Estructura del Texto}
\nextCapitulo{Estructura del Texto}
\nextTema{Partes del texto.}
\nextTema{Identificación de estructura en textos.}

\cite{Real} 


%inicio unidad
\nextUnidad{Estructura de Párrafos}
\nextCapitulo{Estructura de Párrafos}
\nextTema{Partes del párrafo.}
\nextTema{Esquema de párrafos.}

\cite{Real} 


%inicio unidad
\nextUnidad{Características del párrafo}
\nextCapitulo{Características del párrafo}
\nextTema{Ejercicios de redacción de párrafos.}
\nextTema{Indicaciones generales sobre Trabajo Final.}
\nextTema{Co-evaluación sobre párrafo en FORO .}

\cite{Real} 


%inicio unidad
\nextUnidad{Esquema de síntesis}
\nextCapitulo{Esquema de síntesis}
\nextTema{Esquema y Resumen.}

\cite{Real} 


%inicio unidad
\nextUnidad{Texto argumentativo vs. expositivo}
\nextCapitulo{Texto argumentativo vs. expositivo}
\nextTema{Características y partes del texto expositivo.}
\nextTema{Proceso de redacción.}

\cite{Real} 


%inicio unidad
\nextUnidad{Proceso de Redacción}
\nextCapitulo{Proceso de Redacción}
\nextTema{Delimitación de tema y esquema de producción.}
\nextTema{Asesoría de preparación}
\nextTema{Tema, esquema ,producción (partes y subpartes) .}

\cite{Real} 


%inicio unidad
\nextUnidad{Entrega Virtual}
\nextCapitulo{Entrega Virtual}
\nextTema{Tema delimitado.}
\nextTema{Justificación.}
\nextTema{Esquema.}
\nextTema{Reporte de Fuentes.}

\cite{Real} 


%inicio unidad
\nextUnidad{Proceso de Redacción}
\nextCapitulo{Proceso de Redacción}
\nextTema{Función y tipos (APA 6ta edición).}

\cite{Real} 


%inicio unidad
\nextUnidad{Citas}
\nextCapitulo{Citas}
\nextTema{Función y tipos}
\nextTema{Bibliografía}

\cite{Real} 


%inicio unidad
\nextUnidad{Tipos de párrafos}
\nextCapitulo{Tipos de párrafos}
\nextTema{Tipos de párrafos.}
\nextTema{Trabajo grupal en clase.}

\cite{Real} 


%inicio unidad
\nextUnidad{Aproximación a características de la exposición oral}
\nextCapitulo{Aproximación a características de la exposición oral}
\nextTema{Aproximación a características de la exposición oral.}
\nextTema{Ejercicios de escritura}

\cite{Real} 


%inicio unidad
\nextUnidad{Párrafo Comparativo}
\nextCapitulo{Párrafo Comparativo}
\nextTema{Establecimiento de Criterios.}
\nextTema{Asesoría Avance 2}

\cite{Real} 


%inicio unidad
\nextUnidad{Características de la exposición oral y tipos de párrafo}
\nextCapitulo{Características de la exposición oral y tipos de párrafo}
\nextTema{Coevaluaciones: párrafos enumerativo y comparativo.}
\nextTema{FORO: características de oralidad en un contexto académico.}

\cite{Real} 


%inicio unidad
\nextUnidad{Redacción de texto completo}
\nextCapitulo{Redacción de texto completo}
\nextTema{Redacción de texto completo con citas.}

\cite{Real} 


%inicio unidad
\nextUnidad{Teoría: redacción de textos completos}
\nextCapitulo{Teoría: redacción de textos completos}
\nextTema{Asesoría}
\nextTema{Indicaciones para el tercer avance.}

\cite{Real} 


%inicio unidad
\nextUnidad{Exposiciones}
\nextCapitulo{Exposiciones}
\nextTema{Retroalimentación de Exposiciones.}
\nextTema{Coevaluaciones.}

\cite{Real} 





\end{contenidos}




\begin{estrategiasEnsenanza}
    \begin{metodos}
        Método expositivo en las clases teóricas \\
        Método de elaboración conjunta en los seminarios taller y en la elaboración del proyecto de investigación.
    \end{metodos}
    \begin{medios}
        Pizarra acrílica, plumones, cañón multimedia, material de laboratorio, videos, software.
    \end{medios}
    \begin{formasOrganizacion}
        %Se pone los que se necesiten
        \newItemFO{Clases Teóricas}{Desarrollo de los conceptos teóricos}
        \newItemFO{Seminarios}{Algo...}
        \newItemFO{Prácticas}{Algo...}
        \newItemFO{Laboratorio}{Aplicación de los conceptos vistos es clases teóricas.}
        \newItemFO{Otros}{Algo...}
    \end{formasOrganizacion}
    \begin{programacion}
        \newItemFO{Investigación Formativa}{Implementación de Sistema Computacional Web usando una base de datos relacional normalizada}
        \newItemFO{Responsabilidad Social}{Generar videos para la enseñanza de implementación de bases de datos y que sean disponibilizados de la población}
    \end{programacion}
    \begin{segumientoAprendizaje}
        Aquí va el seguimiento del aprendizaje
    \end{segumientoAprendizaje}
\end{estrategiasEnsenanza}


\begin{cronogramaAcademico}
    \newItemCA{Tema1}{Edward Hinojosa Cárdenas}{10}  %Tema/Evaluación - Docente - Porcentaje acumulado
    \newItemCA{Tema2}{Edward Hinojosa Cárdenas}{16}
    \newItemCA{Tema3}{Edward Hinojosa Cárdenas}{20}
    \newItemCA{Tema4}{Edward Hinojosa Cárdenas}{25}
    \newItemCA{Tema5}{Edward Hinojosa Cárdenas}{33}
    \newItemCA{Tema6}{Edward Hinojosa Cárdenas}{37}
    \newItemCA{Tema7}{Edward Hinojosa Cárdenas}{40}
    \newItemCA{Tema8}{Edward Hinojosa Cárdenas}{45}
    \newItemCA{Tema9}{Edward Hinojosa Cárdenas}{50}
    \newItemCA{Tema10}{Edward Hinojosa Cárdenas}{53}
    \newItemCA{Tema11}{Edward Hinojosa Cárdenas}{58}
    \newItemCA{Tema12}{Edward Hinojosa Cárdenas}{64}
    \newItemCA{Tema13}{Edward Hinojosa Cárdenas}{69}
    \newItemCA{Tema14}{Edward Hinojosa Cárdenas}{76}
    \newItemCA{Tema15}{Edward Hinojosa Cárdenas}{84}
    \newItemCA{Tema16}{Edward Hinojosa Cárdenas}{93}
    \newItemCA{Tema17}{Edward Hinojosa Cárdenas}{100}
\end{cronogramaAcademico}

\begin{estrategiasEvaluacion}
    \begin{evaluacionContinua}
        Práctica y Laboratorios en cada clase sobre los temas realizados, tanto para el primer parcial (EC1), segundo parcial (EC2) y tercer parcial (EC3).
    \end{evaluacionContinua}
    \begin{evaluacionPeriodica}
        \newItemEP{Primer Examen}{ponderación}
        \newItemEP{Segundo Examen}{ponderación}
        \newItemEP{Tercer Examen}{}
    \end{evaluacionPeriodica}
    \begin{cronogramaEvaluacion}
        \newItemCE{11/5/2019}{11/5/2019}{11/5/2019}{30\%}
        \newItemCE{11/8/2019}{11/8/2019}{11/8/2019}{30\%}
        \newItemCE{11/12/2019}{11/12/2019}{11/12/2019}{40\%}
    \end{cronogramaEvaluacion}
    \begin{tipoEvaluacion}
        Tipo de evaluación
    \end{tipoEvaluacion}
    \begin{instrumentosEvaluacion}
        Instrumentos de evaluación
    \end{instrumentosEvaluacion}
\end{estrategiasEvaluacion}

\begin{requisitosAprobacion}
\item El alumno tendrá derecho a observar o en su defecto a ratificar las notas consignadas en sus evaluaciones, después de ser entregadas las mismas por parte del profesor, salvo el vencimiento de plazos para culminación del semestre académico, luego del mismo, no se admitirán reclamaciones,
alumno que no se haga presente en el día establecido, perderá su derecho a reclamo.
\item Para aprobar ...
\end{requisitosAprobacion}

\bibliography{FG250.bib}
\bibliographystyle{apalike}

\fecha
\firma

\end{document}


