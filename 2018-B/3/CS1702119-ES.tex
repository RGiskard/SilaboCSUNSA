\documentclass[a4paper,8pt]{article}
\usepackage[utf8]{inputenc}
\usepackage{tabularx}
\usepackage{setspace}
\usepackage{../../silabus}

\begin{document}

%%Setear variables
\setPeriodoAcademico{2018-B}
\setNombreAsignatura{Desarrollo Basado en Plataformas}
\setNombreProfesor{}
\setGradoProfesorAbreviado{}
\sylabusHeader

\academicaTable
{Ciencia de la Computación} %Escuela Profesional
{CS1702119} %Código de la asignatura
{3$^{er}$ Semestre.} %Número del semestre
{Semestral} %Características
{17 Semanas} %Duración
{1 HT} %Número de horas teóricas
{2 HP} %Número de horas prácticas
{0} %Número de horas seminarios
{2 HL}  %Número de horas laboratorio
{2} %Número de horas Teórico-práctico
{3} %Número de créditos
{CS1701209} % Prerrequisitos (separados por comas)

\administrativaTable
{Doctor} %Grado académico del profesor
{Ingeniería de Sistemas e Informática} %Departamento académico
{6} %Número de horas totales
{2} %Número de horas - lunes
{-} %Número de horas - martes
{2} %Número de horas - miercoles
{2} %Número de horas - jueves
{-} %Número de horas - viernes
{101} %Aula de clase - lunes
{-} %Aula de clase - martes
{101} %Aula de clase - miercoles
{101} %Aula de clase - jueves
{-} %Aula de clase - viernes


\begin{fundamentacion}
El mundo ha cambiado debido al uso de la web y tecnologías relacionadas, el acceso rápido, oportuno y personalizado de la
información, a través de la tecnología web, ubícuo  y pervasiva; han cambiado la forma de ?`cómo hacemos las cosas?, ?`cómo pensamos? y ?`cómo la industria se desarrolla?.

Las tecnologías web, ubicuo  y pervasivo se basan en el desarrollo de servicios web, aplicaciones web y aplicaciones móviles,
las cuales son necesarias entender la arquitectura, el diseño, y la implementación de servicios web, aplicaciones web y aplicaciones móviles.

\end{fundamentacion}

\begin{sumilla}
\item \PBDIntroduction
\item \PBDWebPlatforms
\item Desarrollo de servicios y aplicaciones web
\item \PBDMobilePlatforms
\item Aplicaciones Móviles para dispositivos Android

\end{sumilla}

\begin{competenciasAsignatura}
\item \ShowCompetence{C1}{c,d,i}
\item \ShowCompetence{C6}{c,d,i}
\item \ShowCompetence{CS8}{g}

\end{competenciasAsignatura}

\begin{contenidos}


%inicio unidad
\nextUnidad{\PBDIntroduction}
\nextCapitulo{\PBDIntroduction}
\nextTema{\PBDIntroductionTopicOverview}
\nextTema{\PBDIntroductionTopicProgramming}
\nextTema{\PBDIntroductionTopicOverviewOf}
\nextTema{\PBDIntroductionTopicProgrammingUnder}

\cite{grove2009web}, \cite{annuzzi2013introduction} 


%inicio unidad
\nextUnidad{\PBDWebPlatforms}
\nextCapitulo{\PBDWebPlatforms}
\nextTema{\PBDWebPlatformsTopicWeb}
\nextTema{Restricciones de las plataformas web: Client-Server, Stateless-Stateful, Caché, Uniform Interface, Layered System, Code on Demand, ReST.}
\nextTema{\PBDWebPlatformsTopicWebPlatform}
\nextTema{\PBDWebPlatformsTopicSoftware}
\nextTema{\PBDWebPlatformsTopicWebStandards}

\cite{fielding2000fielding} 


%inicio unidad
\nextUnidad{Desarrollo de servicios y aplicaciones web}
\nextCapitulo{Desarrollo de servicios y aplicaciones web}
\nextTema{Describir, identificar y depurar problemas relacionados con el desarrollo de aplicaciones web.}
\nextTema{Diseño y desarrollo de aplicaciones web interactivas usando HTML5 y Python.}
\nextTema{Utilice MySQL para la gestión de datos y manipular MySQL con Python.}
\nextTema{Diseño y desarrollo de aplicaciones web asíncronos utilizando técnicas Ajax.}
\nextTema{Uso del lado del cliente dinámico lenguaje de script Javascript y del lado del servidor lenguaje de scripting python con Ajax.}
\nextTema{Aplicar las tecnologías XML / JSON para la gestión de datos.}
\nextTema{Utilizar los servicios, APIs Web, Ajax y aplicar los patrones de diseño para el desarrollo de aplicaciones web.}

\cite{freeman2011head} 


%inicio unidad
\nextUnidad{\PBDMobilePlatforms}
\nextCapitulo{\PBDMobilePlatforms}
\nextTema{\PBDMobilePlatformsTopicMobile}
\nextTema{Principios de diseño: Segregación de Interfaces, Responsabilidad  Única, Separación de Responsabilidades, Inversión de Dependencias.}
\nextTema{\PBDMobilePlatformsTopicChallenges}
\nextTema{\PBDMobilePlatformsTopicLocation}
\nextTema{\PBDMobilePlatformsTopicPerformance}
\nextTema{\PBDMobilePlatformsTopicMobilePlatform}
\nextTema{\PBDMobilePlatformsTopicEmerging}

\cite{martin2017clean} 


%inicio unidad
\nextUnidad{Aplicaciones Móviles para dispositivos Android}
\nextCapitulo{Aplicaciones Móviles para dispositivos Android}
\nextTema{The Android Platform}
\nextTema{The Android Development Environment}
\nextTema{Application Fundamentals}
\nextTema{The Activity Class}
\nextTema{The Intent Class}
\nextTema{Permissions}
\nextTema{The Fragment Class}
\nextTema{User Interface Classes}
\nextTema{User Notifications}
\nextTema{The BroadcastReceiver Class}
\nextTema{Threads, AsyncTask \& Handlers}
\nextTema{Alarms}
\nextTema{Networking (http class)}
\nextTema{Multi-touch \& Gestures}
\nextTema{Sensors}
\nextTema{Location \& Maps}

\cite{annuzzi2013introduction} 





\end{contenidos}




\begin{estrategiasEnsenanza}
    \begin{metodos}
        Método expositivo en las clases teóricas \\
        Método de elaboración conjunta en los seminarios taller y en la elaboración del proyecto de investigación.
    \end{metodos}
    \begin{medios}
        Pizarra acrílica, plumones, cañón multimedia, material de laboratorio, videos, software.
    \end{medios}
    \begin{formasOrganizacion}
        %Se pone los que se necesiten
        \newItemFO{Clases Teóricas}{Desarrollo de los conceptos teóricos}
        \newItemFO{Seminarios}{Algo...}
        \newItemFO{Prácticas}{Algo...}
        \newItemFO{Laboratorio}{Aplicación de los conceptos vistos es clases teóricas.}
        \newItemFO{Otros}{Algo...}
    \end{formasOrganizacion}
    \begin{programacion}
        \newItemFO{Investigación Formativa}{Implementación de Sistema Computacional Web usando una base de datos relacional normalizada}
        \newItemFO{Responsabilidad Social}{Generar videos para la enseñanza de implementación de bases de datos y que sean disponibilizados de la población}
    \end{programacion}
    \begin{segumientoAprendizaje}
        Aquí va el seguimiento del aprendizaje
    \end{segumientoAprendizaje}
\end{estrategiasEnsenanza}


\begin{cronogramaAcademico}
    \newItemCA{Tema1}{Edward Hinojosa Cárdenas}{10}  %Tema/Evaluación - Docente - Porcentaje acumulado
    \newItemCA{Tema2}{Edward Hinojosa Cárdenas}{16}
    \newItemCA{Tema3}{Edward Hinojosa Cárdenas}{20}
    \newItemCA{Tema4}{Edward Hinojosa Cárdenas}{25}
    \newItemCA{Tema5}{Edward Hinojosa Cárdenas}{33}
    \newItemCA{Tema6}{Edward Hinojosa Cárdenas}{37}
    \newItemCA{Tema7}{Edward Hinojosa Cárdenas}{40}
    \newItemCA{Tema8}{Edward Hinojosa Cárdenas}{45}
    \newItemCA{Tema9}{Edward Hinojosa Cárdenas}{50}
    \newItemCA{Tema10}{Edward Hinojosa Cárdenas}{53}
    \newItemCA{Tema11}{Edward Hinojosa Cárdenas}{58}
    \newItemCA{Tema12}{Edward Hinojosa Cárdenas}{64}
    \newItemCA{Tema13}{Edward Hinojosa Cárdenas}{69}
    \newItemCA{Tema14}{Edward Hinojosa Cárdenas}{76}
    \newItemCA{Tema15}{Edward Hinojosa Cárdenas}{84}
    \newItemCA{Tema16}{Edward Hinojosa Cárdenas}{93}
    \newItemCA{Tema17}{Edward Hinojosa Cárdenas}{100}
\end{cronogramaAcademico}

\begin{estrategiasEvaluacion}
    \begin{evaluacionContinua}
        Práctica y Laboratorios en cada clase sobre los temas realizados, tanto para el primer parcial (EC1), segundo parcial (EC2) y tercer parcial (EC3).
    \end{evaluacionContinua}
    \begin{evaluacionPeriodica}
        \newItemEP{Primer Examen}{ponderación}
        \newItemEP{Segundo Examen}{ponderación}
        \newItemEP{Tercer Examen}{}
    \end{evaluacionPeriodica}
    \begin{cronogramaEvaluacion}
        \newItemCE{11/5/2019}{11/5/2019}{11/5/2019}{30\%}
        \newItemCE{11/8/2019}{11/8/2019}{11/8/2019}{30\%}
        \newItemCE{11/12/2019}{11/12/2019}{11/12/2019}{40\%}
    \end{cronogramaEvaluacion}
    \begin{tipoEvaluacion}
        Tipo de evaluación
    \end{tipoEvaluacion}
    \begin{instrumentosEvaluacion}
        Instrumentos de evaluación
    \end{instrumentosEvaluacion}
\end{estrategiasEvaluacion}

\begin{requisitosAprobacion}
\item El alumno tendrá derecho a observar o en su defecto a ratificar las notas consignadas en sus evaluaciones, después de ser entregadas las mismas por parte del profesor, salvo el vencimiento de plazos para culminación del semestre académico, luego del mismo, no se admitirán reclamaciones,
alumno que no se haga presente en el día establecido, perderá su derecho a reclamo.
\item Para aprobar ...
\end{requisitosAprobacion}

\bibliography{CS2B1.bib}
\bibliographystyle{apalike}

\fecha
\firma

\end{document}


