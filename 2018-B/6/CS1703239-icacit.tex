\documentclass[12pt]{article}
\usepackage[utf8]{inputenc}
\usepackage{vmargin}
\usepackage{fancyhdr}
\usepackage{graphics}
\usepackage{../../icacit}
\usepackage[margin=2.5cm,headheight=200pt,includeheadfoot]{geometry}

\setpapersize{A4}
\setmargins{2.5cm}       % margen izquierdo
{0.5cm}                        % margen superior
{16.5cm}                      % anchura del texto
{23.42cm}                    % altura del texto
{10pt}                           % altura de los encabezados
{1cm}                           % espacio entre el texto y los encabezados
{0pt}                             % altura del pie de página
{2cm}                           % espacio entre el texto y el pie de página


\begin{document}

\sylabusHeader
\sylabusTitle

\curso
{1003239} %Código del curso
{SISTEMAS OPERATIVOS} %Nombre del curso
{2018-B} %Semestre

\creditosHoras
{5} %Número de creditos
{2} %Horas teoría
{2} %Horas práctica
{} %Horas teórico-practica
{2} %Horas laboratorio
{6} %Total de horas

\instructor
{Dra. Roxana Flores Quispe}

%\libro
%{Operating System Concepts, (9th ed.)} %Título del libro
%{Avi Silberschatz, Peter Baer Galvin, Greg Gagne} %Autor %del libro
%{2012} %Año del libro


\libro
{Operating Systems: Internals and Design Principles, (5th ed.)} %Título del libro
{William Stallings} %Autor del libro
{2005} %Año del libro

%\libro



\libroSecundario
{Operating Systems Design and Implementation, (3th ed.)} %Título del libro
{Andrew S. Tanenbaum} %Autor del libro
{2006} %Año del libro
%{Apuntes de Sistemas Operativos} %Título del libro
%{Luis Mateu} %Autor del libro
%{1999} %Año del libro

\begin{datosCurso}
    \begin{descripcion}
      Este curso presenta los conceptos fundamentales de los sistemas operativos actuales. Se introducen conceptos de planificación,  programación  concurrente  y  se analizan  problemas  que  los  mismos  presentan  como:  exclusión  mutua, interbloqueo  e  inanición.  Se  analizan  también  las  estructuras  de  datos  usadas  para  administrar  la  memoria principal  y  el  sistema  de  archivos  en  un  sistema  operativo  real.  A  fin  de  validar  estos  conceptos.
      %,  el  curso incluye  prácticas  y  proyectos  donde  el  estudiante  debe  diseñar  e  implementar  soluciones  a  problemas  que requieren explotar las funcionalidades de un sistema operativo%
      

        
    \end{descripcion}
    \begin{requisitos}
        \newItemReq{1702117}{Arquitectura de Computadores}
        
    \end{requisitos}
    \ObligatorioElectivo
    {X} %Marcar esta si es obligatorio 
    {} %Marcar esta si es electivo
\end{datosCurso}

\begin{objetivosCurso}
    \begin{resultadosEspecificos}
    Este curso tiene por objetivo que el alumno pueda analizar y comprender el funcionamiento de cómo se administran los recursos software y hardware del computador incluyendo la planificación de procesos, la gestión de memoria, sistema de archivos y disco, para que el alumno aplique su pensamiento crítico, creativo e innovador al resolver problemas de su entorno de acuerdo al análisis  respectivo.
    
       
    \end{resultadosEspecificos}
    \begin{resultadosEstudiante}
        1) Aplica los fundamentos matemáticos, principios algorítmicos y teoría de ciencias de la computación en el modelamiento y diseño de sistemas basados en computadora para simular y optimizar la planificación y gestión de procesos, memoria y disco.  2) Aplica en un nivel intermedio.
    \end{resultadosEstudiante}
\end{objetivosCurso}

\begin{ListaTemas}
%\nextUnidad{FUNDAMENTOS DE LOS SISTEMAS OPERATIVOS} 
\nextUnidad{PROCESOS E HILOS} 
\nextUnidad{GESTOR DE ENTRADA / SALIDA} 
\nextUnidad{GESTOR DE MEMORIA Y SISTEMA DE ARCHIVOS} 

\end{ListaTemas}


\end{document}