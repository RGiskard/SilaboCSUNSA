\documentclass[a4paper,8pt]{article}
\usepackage[utf8]{inputenc}
\usepackage{tabularx}
\usepackage{setspace}
\usepackage{../../silabus}

\begin{document}

%%Setear variables
\setPeriodoAcademico{2018-B}
\setNombreAsignatura{Formación de Empresas de Base Tecnológica II}
\setNombreProfesor{}
\setGradoProfesorAbreviado{}
\sylabusHeader

\academicaTable
{Ciencia de la Computación} %Escuela Profesional
{ET1705159} %Código de la asignatura
{9$^{no}$ Semestre.} %Número del semestre
{Semestral} %Características
{17 Semanas} %Duración
{1 HT} %Número de horas teóricas
{2 HP} %Número de horas prácticas
{0} %Número de horas seminarios
{}  %Número de horas laboratorio
{2} %Número de horas Teórico-práctico
{2} %Número de créditos
{ET1704252} % Prerrequisitos (separados por comas)

\administrativaTable
{Doctor} %Grado académico del profesor
{Ingeniería de Sistemas e Informática} %Departamento académico
{6} %Número de horas totales
{2} %Número de horas - lunes
{-} %Número de horas - martes
{2} %Número de horas - miercoles
{2} %Número de horas - jueves
{-} %Número de horas - viernes
{101} %Aula de clase - lunes
{-} %Aula de clase - martes
{101} %Aula de clase - miercoles
{101} %Aula de clase - jueves
{-} %Aula de clase - viernes


\begin{fundamentacion}
Este curso tiene como objetivo dotar al futuro profesional de conocimientos, actitudes y aptitudes que le permitan formar su propia empresa de desarrollo de software y/o consultoría en informática. El curso está dividido en tres unidades: Valorización de Proyectos, Marketing de Servicios y Negociaciones. En la primera unidad se busca que el alumno pueda analizar y tomar decisiones en relación a la viabilidad de un proyecto y/o negocio.

En la segunda unidad se busca preparar al alumno para que este pueda llevar a cabo un plan de marketing satisfactorio del bien o servicio que su empresa pueda ofrecer al mercado. La tercera unidad busca desarrollar la capacidad negociadora de los participantes a través del entrenamiento vivencial y práctico y de los conocimientos teóricos que le permitan cerrar contrataciones donde tanto el cliente como el proveedor resulten ganadores. Consideramos estos temas sumamente críticos en las etapas de lanzamiento, consolidación y eventual relanzamiento de una empresa de base tecnológica.

\end{fundamentacion}

\begin{sumilla}
\item 
\item 
\item 

\end{sumilla}

\begin{competenciasAsignatura}
\item \ShowCompetence{C17}{f}
\item \ShowCompetence{C18}{d}
\item \ShowCompetence{C19}{m}
\item \ShowCompetence{C20}{m}
\item \ShowCompetence{C21}{m}
\item \ShowCompetence{C22}{m}
\item \ShowCompetence{C23}{m}
\item \ShowCompetence{C24}{m}

\end{competenciasAsignatura}

\begin{contenidos}


%inicio unidad
\nextUnidad{}
\nextCapitulo{}
\nextTema{Introducción}
\nextTema{Proceso de toma de decisiones}
\nextTema{El valor del dinero en el tiempo}
\nextTema{Tasa de interés y tasa de rendimiento}
\nextTema{Interés simple e interés compuesto}
\nextTema{Identificación de costos}
\nextTema{Flujo de Caja Neto}
\nextTema{Tasa de Retorno de Inversión (TIR)}
\nextTema{Valor Presente Neto (VPN)}
\nextTema{Valorización de Proyectos}

\cite{blank06} 


%inicio unidad
\nextUnidad{}
\nextCapitulo{}
\nextTema{Introducción}
\nextTema{Importancia del marketing en las empresas de servicios}
\nextTema{El Proceso estratégico.}
\nextTema{El Plan de Marketing}
\nextTema{Marketing estratégico y marketing operativo}
\nextTema{Segmentación, targeting y posicionamiento de servicios en mercados competitivos}
\nextTema{Ciclo de vida del producto}
\nextTema{Aspectos a considerar en la fijación de precios en servicios}
\nextTema{El rol de la publicidad, las ventas y otras formas de comunicación}
\nextTema{El comportamiento del consumidor en servicios}
\nextTema{Fundamentos de marketing de servicios}
\nextTema{Creación del modelo de servicio}
\nextTema{Gestión de la calidad de servicio}

\cite{kotler06}, \cite{love09} 


%inicio unidad
\nextUnidad{}
\nextCapitulo{}
\nextTema{Introducción. ?`Qué es una negociación?}
\nextTema{Teoría de las necesidades de la negociación}
\nextTema{La proceso de la negociación}
\nextTema{Estilos de negociación}
\nextTema{Teoría de juegos}
\nextTema{El método Harvard de negociación}

\cite{fish96}, \cite{dasi06} 





\end{contenidos}




\begin{estrategiasEnsenanza}
    \begin{metodos}
        Método expositivo en las clases teóricas \\
        Método de elaboración conjunta en los seminarios taller y en la elaboración del proyecto de investigación.
    \end{metodos}
    \begin{medios}
        Pizarra acrílica, plumones, cañón multimedia, material de laboratorio, videos, software.
    \end{medios}
    \begin{formasOrganizacion}
        %Se pone los que se necesiten
        \newItemFO{Clases Teóricas}{Desarrollo de los conceptos teóricos}
        \newItemFO{Seminarios}{Algo...}
        \newItemFO{Prácticas}{Algo...}
        \newItemFO{Laboratorio}{Aplicación de los conceptos vistos es clases teóricas.}
        \newItemFO{Otros}{Algo...}
    \end{formasOrganizacion}
    \begin{programacion}
        \newItemFO{Investigación Formativa}{Implementación de Sistema Computacional Web usando una base de datos relacional normalizada}
        \newItemFO{Responsabilidad Social}{Generar videos para la enseñanza de implementación de bases de datos y que sean disponibilizados de la población}
    \end{programacion}
    \begin{segumientoAprendizaje}
        Aquí va el seguimiento del aprendizaje
    \end{segumientoAprendizaje}
\end{estrategiasEnsenanza}


\begin{cronogramaAcademico}
    \newItemCA{Tema1}{Edward Hinojosa Cárdenas}{10}  %Tema/Evaluación - Docente - Porcentaje acumulado
    \newItemCA{Tema2}{Edward Hinojosa Cárdenas}{16}
    \newItemCA{Tema3}{Edward Hinojosa Cárdenas}{20}
    \newItemCA{Tema4}{Edward Hinojosa Cárdenas}{25}
    \newItemCA{Tema5}{Edward Hinojosa Cárdenas}{33}
    \newItemCA{Tema6}{Edward Hinojosa Cárdenas}{37}
    \newItemCA{Tema7}{Edward Hinojosa Cárdenas}{40}
    \newItemCA{Tema8}{Edward Hinojosa Cárdenas}{45}
    \newItemCA{Tema9}{Edward Hinojosa Cárdenas}{50}
    \newItemCA{Tema10}{Edward Hinojosa Cárdenas}{53}
    \newItemCA{Tema11}{Edward Hinojosa Cárdenas}{58}
    \newItemCA{Tema12}{Edward Hinojosa Cárdenas}{64}
    \newItemCA{Tema13}{Edward Hinojosa Cárdenas}{69}
    \newItemCA{Tema14}{Edward Hinojosa Cárdenas}{76}
    \newItemCA{Tema15}{Edward Hinojosa Cárdenas}{84}
    \newItemCA{Tema16}{Edward Hinojosa Cárdenas}{93}
    \newItemCA{Tema17}{Edward Hinojosa Cárdenas}{100}
\end{cronogramaAcademico}

\begin{estrategiasEvaluacion}
    \begin{evaluacionContinua}
        Práctica y Laboratorios en cada clase sobre los temas realizados, tanto para el primer parcial (EC1), segundo parcial (EC2) y tercer parcial (EC3).
    \end{evaluacionContinua}
    \begin{evaluacionPeriodica}
        \newItemEP{Primer Examen}{ponderación}
        \newItemEP{Segundo Examen}{ponderación}
        \newItemEP{Tercer Examen}{}
    \end{evaluacionPeriodica}
    \begin{cronogramaEvaluacion}
        \newItemCE{11/5/2019}{11/5/2019}{11/5/2019}{30\%}
        \newItemCE{11/8/2019}{11/8/2019}{11/8/2019}{30\%}
        \newItemCE{11/12/2019}{11/12/2019}{11/12/2019}{40\%}
    \end{cronogramaEvaluacion}
    \begin{tipoEvaluacion}
        Tipo de evaluación
    \end{tipoEvaluacion}
    \begin{instrumentosEvaluacion}
        Instrumentos de evaluación
    \end{instrumentosEvaluacion}
\end{estrategiasEvaluacion}

\begin{requisitosAprobacion}
\item El alumno tendrá derecho a observar o en su defecto a ratificar las notas consignadas en sus evaluaciones, después de ser entregadas las mismas por parte del profesor, salvo el vencimiento de plazos para culminación del semestre académico, luego del mismo, no se admitirán reclamaciones,
alumno que no se haga presente en el día establecido, perderá su derecho a reclamo.
\item Para aprobar ...
\end{requisitosAprobacion}

\bibliography{ET301.bib}
\bibliographystyle{apalike}

\fecha
\firma

\end{document}


