\documentclass[a4paper,8pt]{article}
\usepackage[utf8]{inputenc}
\usepackage{tabularx}
\usepackage{setspace}
\usepackage{../../silabus}

\begin{document}

%%Setear variables
\setPeriodoAcademico{2018-B}
\setNombreAsignatura{Lengua Extranjera III}
\setNombreProfesor{}
\setGradoProfesorAbreviado{}
\sylabusHeader

\academicaTable
{Ciencia de la Computación} %Escuela Profesional
{ID1002118} %Código de la asignatura
{3$^{er}$ Semestre.} %Número del semestre
{Semestral} %Características
{17 Semanas} %Duración
{2 HT} %Número de horas teóricas
{2 HP} %Número de horas prácticas
{0} %Número de horas seminarios
{}  %Número de horas laboratorio
{2} %Número de horas Teórico-práctico
{3} %Número de créditos
{ID1001212} % Prerrequisitos (separados por comas)

\administrativaTable
{Doctor} %Grado académico del profesor
{Ingeniería de Sistemas e Informática} %Departamento académico
{6} %Número de horas totales
{2} %Número de horas - lunes
{-} %Número de horas - martes
{2} %Número de horas - miercoles
{2} %Número de horas - jueves
{-} %Número de horas - viernes
{101} %Aula de clase - lunes
{-} %Aula de clase - martes
{101} %Aula de clase - miercoles
{101} %Aula de clase - jueves
{-} %Aula de clase - viernes


\begin{fundamentacion}
Parte fundamental de la formación integral de un profesional es la habilidad de comunicarse en un idioma extranjero además del propio idioma nativo. No solamente amplía su horizonte cultural sino que permite una visión más humana y comprensiva de la vida. En el caso de los idiomas extranjeros, indudablemente el Inglés es el 
más práctico porque es hablado alrededor de todo el mundo. No hay país alguno donde este no sea hablado. En las carreras relacionadas con los servicios al turista el inglés es tal vez la herramienta práctica más importante que el alumno debe dominar desde el primer momento como parte de su formación integral.

\end{fundamentacion}

\begin{sumilla}
\item Getting to know you!
\item The way we live!
\item It all went wrong!
\item Lets go shopping!
\item What do you want to do?
\item The best in the world!
\item Fame!

\end{sumilla}

\begin{competenciasAsignatura}
\item \ShowCompetence{C25}{f}

\end{competenciasAsignatura}

\begin{contenidos}


%inicio unidad
\nextUnidad{Getting to know you!}
\nextCapitulo{Getting to know you!}
\nextTema{Tiempos Presente, Pasado y Futuro.}
\nextTema{Oraciones Interrogativas con Wh-.}
\nextTema{Palabras con más de un significado.}
\nextTema{Partes de la oración.}
\nextTema{Expresiones para tiempo libre.}

\cite{Soars022S}, \cite{Cambridge06}, \cite{MacGrew99} 


%inicio unidad
\nextUnidad{The way we live!}
\nextCapitulo{The way we live!}
\nextTema{Tiempo Presente Simple.}
\nextTema{Tiempo Presente Continuo.}
\nextTema{Colocaciones.}
\nextTema{Vocabulario de paí­ses del  mundo.}
\nextTema{Expresiones de enojo.}
\nextTema{Conectores.}

\cite{Soars022S}, \cite{Cambridge06}, \cite{MacGrew99} 


%inicio unidad
\nextUnidad{It all went wrong!}
\nextCapitulo{It all went wrong!}
\nextTema{Tiempo Pasado Simple.}
\nextTema{Tiempo Pasado Continuo.}
\nextTema{Verbos Irregulares.}
\nextTema{Expresiones de Tiempo.}
\nextTema{Conectores de tiempo.}

\cite{Soars022S}, \cite{Cambridge06}, \cite{MacGrew99} 


%inicio unidad
\nextUnidad{Lets go shopping!}
\nextCapitulo{Lets go shopping!}
\nextTema{Expresiones de Cantidad Indefinida}
\nextTema{Oraciones Afirmativas, Negativas y Preguntas}
\nextTema{Uso de Artí­culos}
\nextTema{Precios de productos}
\nextTema{Llenado de formatos y encuestas}
\nextTema{Expresiones para ir de compras}

\cite{Soars022S}, \cite{Cambridge06}, \cite{MacGrew99} 


%inicio unidad
\nextUnidad{What do you want to do?}
\nextCapitulo{What do you want to do?}
\nextTema{Patrones Verbales I.}
\nextTema{Intenciones Futuras.}
\nextTema{Verbos de Percepción.}
\nextTema{Vocabulario de sentimientos.}
\nextTema{Expresiones de Planes y Ambiciones.}

\cite{Soars022S}, \cite{Cambridge06}, \cite{MacGrew99} 


%inicio unidad
\nextUnidad{The best in the world!}
\nextCapitulo{The best in the world!}
\nextTema{Whats it like?.}
\nextTema{Adjetivos.}
\nextTema{Comparativos y Superlativos.}
\nextTema{Sinónimos y Antónimos.}
\nextTema{Indicaciones de Dirección.}
\nextTema{Lecturas.}

\cite{Soars022S}, \cite{Cambridge06}, \cite{MacGrew99} 


%inicio unidad
\nextUnidad{Fame!}
\nextCapitulo{Fame!}
\nextTema{Presente Perfecto y Pasado Simple}
\nextTema{Expresiones for, ever, since}
\nextTema{Adverbios}
\nextTema{Expresiones que vienen en pares}
\nextTema{Respuestas cortas}
\nextTema{Celebridades}

\cite{Soars022S}, \cite{Cambridge06}, \cite{MacGrew99} 





\end{contenidos}




\begin{estrategiasEnsenanza}
    \begin{metodos}
        Método expositivo en las clases teóricas \\
        Método de elaboración conjunta en los seminarios taller y en la elaboración del proyecto de investigación.
    \end{metodos}
    \begin{medios}
        Pizarra acrílica, plumones, cañón multimedia, material de laboratorio, videos, software.
    \end{medios}
    \begin{formasOrganizacion}
        %Se pone los que se necesiten
        \newItemFO{Clases Teóricas}{Desarrollo de los conceptos teóricos}
        \newItemFO{Seminarios}{Algo...}
        \newItemFO{Prácticas}{Algo...}
        \newItemFO{Laboratorio}{Aplicación de los conceptos vistos es clases teóricas.}
        \newItemFO{Otros}{Algo...}
    \end{formasOrganizacion}
    \begin{programacion}
        \newItemFO{Investigación Formativa}{Implementación de Sistema Computacional Web usando una base de datos relacional normalizada}
        \newItemFO{Responsabilidad Social}{Generar videos para la enseñanza de implementación de bases de datos y que sean disponibilizados de la población}
    \end{programacion}
    \begin{segumientoAprendizaje}
        Aquí va el seguimiento del aprendizaje
    \end{segumientoAprendizaje}
\end{estrategiasEnsenanza}


\begin{cronogramaAcademico}
    \newItemCA{Tema1}{Edward Hinojosa Cárdenas}{10}  %Tema/Evaluación - Docente - Porcentaje acumulado
    \newItemCA{Tema2}{Edward Hinojosa Cárdenas}{16}
    \newItemCA{Tema3}{Edward Hinojosa Cárdenas}{20}
    \newItemCA{Tema4}{Edward Hinojosa Cárdenas}{25}
    \newItemCA{Tema5}{Edward Hinojosa Cárdenas}{33}
    \newItemCA{Tema6}{Edward Hinojosa Cárdenas}{37}
    \newItemCA{Tema7}{Edward Hinojosa Cárdenas}{40}
    \newItemCA{Tema8}{Edward Hinojosa Cárdenas}{45}
    \newItemCA{Tema9}{Edward Hinojosa Cárdenas}{50}
    \newItemCA{Tema10}{Edward Hinojosa Cárdenas}{53}
    \newItemCA{Tema11}{Edward Hinojosa Cárdenas}{58}
    \newItemCA{Tema12}{Edward Hinojosa Cárdenas}{64}
    \newItemCA{Tema13}{Edward Hinojosa Cárdenas}{69}
    \newItemCA{Tema14}{Edward Hinojosa Cárdenas}{76}
    \newItemCA{Tema15}{Edward Hinojosa Cárdenas}{84}
    \newItemCA{Tema16}{Edward Hinojosa Cárdenas}{93}
    \newItemCA{Tema17}{Edward Hinojosa Cárdenas}{100}
\end{cronogramaAcademico}

\begin{estrategiasEvaluacion}
    \begin{evaluacionContinua}
        Práctica y Laboratorios en cada clase sobre los temas realizados, tanto para el primer parcial (EC1), segundo parcial (EC2) y tercer parcial (EC3).
    \end{evaluacionContinua}
    \begin{evaluacionPeriodica}
        \newItemEP{Primer Examen}{ponderación}
        \newItemEP{Segundo Examen}{ponderación}
        \newItemEP{Tercer Examen}{}
    \end{evaluacionPeriodica}
    \begin{cronogramaEvaluacion}
        \newItemCE{11/5/2019}{11/5/2019}{11/5/2019}{30\%}
        \newItemCE{11/8/2019}{11/8/2019}{11/8/2019}{30\%}
        \newItemCE{11/12/2019}{11/12/2019}{11/12/2019}{40\%}
    \end{cronogramaEvaluacion}
    \begin{tipoEvaluacion}
        Tipo de evaluación
    \end{tipoEvaluacion}
    \begin{instrumentosEvaluacion}
        Instrumentos de evaluación
    \end{instrumentosEvaluacion}
\end{estrategiasEvaluacion}

\begin{requisitosAprobacion}
\item El alumno tendrá derecho a observar o en su defecto a ratificar las notas consignadas en sus evaluaciones, después de ser entregadas las mismas por parte del profesor, salvo el vencimiento de plazos para culminación del semestre académico, luego del mismo, no se admitirán reclamaciones,
alumno que no se haga presente en el día establecido, perderá su derecho a reclamo.
\item Para aprobar ...
\end{requisitosAprobacion}

\bibliography{ID101.bib}
\bibliographystyle{apalike}

\fecha
\firma

\end{document}


