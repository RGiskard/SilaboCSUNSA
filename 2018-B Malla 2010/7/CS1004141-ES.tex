\documentclass[a4paper,8pt]{article}
\usepackage[utf8]{inputenc}
\usepackage{tabularx}
\usepackage{setspace}
\usepackage{../../silabus}

\begin{document}

%%Setear variables
\setPeriodoAcademico{2018-B}
\setNombreAsignatura{Lenguajes de Programación}
\setNombreProfesor{}
\setGradoProfesorAbreviado{}
\sylabusHeader

\academicaTable
{Ciencia de la Computación} %Escuela Profesional
{CS1004141} %Código de la asignatura
{7$^{mo}$ Semestre.} %Número del semestre
{Semestral} %Características
{17 Semanas} %Duración
{2 HT} %Número de horas teóricas
{2 HP} %Número de horas prácticas
{0} %Número de horas seminarios
{2 HL}  %Número de horas laboratorio
{2} %Número de horas Teórico-práctico
{4} %Número de créditos
{CS1002219,CS1003124} % Prerrequisitos (separados por comas)

\administrativaTable
{Doctor} %Grado académico del profesor
{Ingeniería de Sistemas e Informática} %Departamento académico
{6} %Número de horas totales
{2} %Número de horas - lunes
{-} %Número de horas - martes
{2} %Número de horas - miercoles
{2} %Número de horas - jueves
{-} %Número de horas - viernes
{101} %Aula de clase - lunes
{-} %Aula de clase - martes
{101} %Aula de clase - miercoles
{101} %Aula de clase - jueves
{-} %Aula de clase - viernes


\begin{fundamentacion}
Los lenguajes de programación son el medio a través del cual los programadores describen con precisión los conceptos, 
formulan algoritmos y representan sus soluciones. Un científico de la computación trabajará con diferentes lenguajes, 
por separado o en conjunto. Los científicos de la computación deben entender los modelos de programación de los diferentes 
lenguajes, tomar decisiones de diseño basados en el lenguaje de programación y sus conceptos. El profesional a menudo 
necesitará aprender nuevos lenguajes y construcciones de programación y debe entender los fundamentos de como las 
características del lenguaje de programación estan definidas, compuestas e implementadas. El uso eficaz de los lenguajes 
de programación y la apreciación de sus limitaciones, también requiere un conocimiento básico de traducción de lenguajes 
de programación y su análisis de ambientes estáticos y dinámicos, así como los componentes de tiempo de ejecución tales 
como la gestión de memoria, entre otros detalles de relevancia.

\end{fundamentacion}

\begin{sumilla}
\item 
\item \PLLanguagePragmatics
\item \PLTypeSystems
\item \PLObjectOrientedProgramming
\item \PLFunctionalProgramming
\item \PLEventDrivenandReactiveProgramming
\item \PLLogicProgramming

\end{sumilla}

\begin{competenciasAsignatura}
\item \ShowCompetence{C2}{a}
\item \ShowCompetence{CS1}{b}
\item \ShowCompetence{CS2}{i}
\item \ShowCompetence{CS3}{j}
\item \ShowCompetence{CS6}{j}

\end{competenciasAsignatura}

\begin{contenidos}


%inicio unidad
\nextUnidad{}
\nextCapitulo{}
\nextTema{Historia de los Lenguajes de Programación}
\nextTema{\PLProgramRepresentationTopicPrograms}
\nextTema{\PLProgramRepresentationTopicData}
\nextTema{Estructura de un programa: Léxico, Sintáctico y Semántico}
\nextTema{BNF}
\nextTema{\PLLanguageTranslationandExecutionTopicInterpretation [\Familiarity]}

\cite{robsebesta}, \cite{webber} 


%inicio unidad
\nextUnidad{\PLLanguagePragmatics}
\nextCapitulo{\PLLanguagePragmatics}
\nextTema{\PLLanguagePragmaticsTopicPrinciples}
\nextTema{\PLLanguagePragmaticsTopicEvaluation}
\nextTema{\PLLanguagePragmaticsTopicEager}
\nextTema{\PLLanguagePragmaticsTopicDefining}
\nextTema{\PLLanguagePragmaticsTopicExternal}

\cite{robsebesta}, \cite{webber}, \cite{vanroy} 


%inicio unidad
\nextUnidad{\PLTypeSystems}
\nextCapitulo{\PLTypeSystems}
\nextTema{\PLTypeSystemsTopicCompositional}
\nextTema{\PLTypeSystemsTopicType}
\nextTema{\PLTypeSystemsTopicTypeSafety}
\nextTema{\PLTypeSystemsTopicTypeInference}
\nextTema{\PLTypeSystemsTopicStatic}

\cite{robsebesta}, \cite{webber}, \cite{vanroy} 


%inicio unidad
\nextUnidad{\PLObjectOrientedProgramming}
\nextCapitulo{\PLObjectOrientedProgramming}
\nextTema{\PLObjectOrientedProgrammingTopicObject}
\nextTema{\PLObjectOrientedProgrammingTopicDefinition}
\nextTema{\PLObjectOrientedProgrammingTopicSubclasses}
\nextTema{\PLObjectOrientedProgrammingTopicDynamic}
\nextTema{\PLObjectOrientedProgrammingTopicSubtyping}
\nextTema{\PLObjectOrientedProgrammingTopicObjectOriented}
\nextTema{\PLObjectOrientedProgrammingTopicUsing}

\cite{robsebesta}, \cite{webber}, \cite{vanroy} 


%inicio unidad
\nextUnidad{\PLFunctionalProgramming}
\nextCapitulo{\PLFunctionalProgramming}
\nextTema{\PLFunctionalProgrammingTopicEffect}
\nextTema{\PLFunctionalProgrammingTopicProcessing}
\nextTema{\PLFunctionalProgrammingTopicFirst}
\nextTema{\PLFunctionalProgrammingTopicFunction}
\nextTema{\PLFunctionalProgrammingTopicDefining}

\cite{robsebesta}, \cite{webber}, \cite{vanroy} 


%inicio unidad
\nextUnidad{\PLEventDrivenandReactiveProgramming}
\nextCapitulo{\PLEventDrivenandReactiveProgramming}
\nextTema{\PLEventDrivenandReactiveProgrammingTopicEvents%}
\nextTema{\PLEventDrivenandReactiveProgrammingTopicCanonical%}
\nextTema{\PLEventDrivenandReactiveProgrammingTopicUsingA%}
\nextTema{\PLEventDrivenandReactiveProgrammingTopicExternally%}
\nextTema{\PLEventDrivenandReactiveProgrammingTopicSeparation%}

\cite{robsebesta} 


%inicio unidad
\nextUnidad{\PLLogicProgramming}
\nextCapitulo{\PLLogicProgramming}
\nextTema{\PLLogicProgrammingTopicCausal%}
\nextTema{\PLLogicProgrammingTopicUnification%}
\nextTema{\PLLogicProgrammingTopicBactracking%}
\nextTema{\PLLogicProgrammingTopicCuts%}

\cite{robsebesta}, \cite{webber}, \cite{vanroy} 





\end{contenidos}




\begin{estrategiasEnsenanza}
    \begin{metodos}
        Método expositivo en las clases teóricas \\
        Método de elaboración conjunta en los seminarios taller y en la elaboración del proyecto de investigación.
    \end{metodos}
    \begin{medios}
        Pizarra acrílica, plumones, cañón multimedia, material de laboratorio, videos, software.
    \end{medios}
    \begin{formasOrganizacion}
        %Se pone los que se necesiten
        \newItemFO{Clases Teóricas}{Desarrollo de los conceptos teóricos}
        \newItemFO{Seminarios}{Algo...}
        \newItemFO{Prácticas}{Algo...}
        \newItemFO{Laboratorio}{Aplicación de los conceptos vistos es clases teóricas.}
        \newItemFO{Otros}{Algo...}
    \end{formasOrganizacion}
    \begin{programacion}
        \newItemFO{Investigación Formativa}{Implementación de Sistema Computacional Web usando una base de datos relacional normalizada}
        \newItemFO{Responsabilidad Social}{Generar videos para la enseñanza de implementación de bases de datos y que sean disponibilizados de la población}
    \end{programacion}
    \begin{segumientoAprendizaje}
        Aquí va el seguimiento del aprendizaje
    \end{segumientoAprendizaje}
\end{estrategiasEnsenanza}


\begin{cronogramaAcademico}
    \newItemCA{Tema1}{Edward Hinojosa Cárdenas}{10}  %Tema/Evaluación - Docente - Porcentaje acumulado
    \newItemCA{Tema2}{Edward Hinojosa Cárdenas}{16}
    \newItemCA{Tema3}{Edward Hinojosa Cárdenas}{20}
    \newItemCA{Tema4}{Edward Hinojosa Cárdenas}{25}
    \newItemCA{Tema5}{Edward Hinojosa Cárdenas}{33}
    \newItemCA{Tema6}{Edward Hinojosa Cárdenas}{37}
    \newItemCA{Tema7}{Edward Hinojosa Cárdenas}{40}
    \newItemCA{Tema8}{Edward Hinojosa Cárdenas}{45}
    \newItemCA{Tema9}{Edward Hinojosa Cárdenas}{50}
    \newItemCA{Tema10}{Edward Hinojosa Cárdenas}{53}
    \newItemCA{Tema11}{Edward Hinojosa Cárdenas}{58}
    \newItemCA{Tema12}{Edward Hinojosa Cárdenas}{64}
    \newItemCA{Tema13}{Edward Hinojosa Cárdenas}{69}
    \newItemCA{Tema14}{Edward Hinojosa Cárdenas}{76}
    \newItemCA{Tema15}{Edward Hinojosa Cárdenas}{84}
    \newItemCA{Tema16}{Edward Hinojosa Cárdenas}{93}
    \newItemCA{Tema17}{Edward Hinojosa Cárdenas}{100}
\end{cronogramaAcademico}

\begin{estrategiasEvaluacion}
    \begin{evaluacionContinua}
        Práctica y Laboratorios en cada clase sobre los temas realizados, tanto para el primer parcial (EC1), segundo parcial (EC2) y tercer parcial (EC3).
    \end{evaluacionContinua}
    \begin{evaluacionPeriodica}
        \newItemEP{Primer Examen}{ponderación}
        \newItemEP{Segundo Examen}{ponderación}
        \newItemEP{Tercer Examen}{}
    \end{evaluacionPeriodica}
    \begin{cronogramaEvaluacion}
        \newItemCE{11/5/2019}{11/5/2019}{11/5/2019}{30\%}
        \newItemCE{11/8/2019}{11/8/2019}{11/8/2019}{30\%}
        \newItemCE{11/12/2019}{11/12/2019}{11/12/2019}{40\%}
    \end{cronogramaEvaluacion}
    \begin{tipoEvaluacion}
        Tipo de evaluación
    \end{tipoEvaluacion}
    \begin{instrumentosEvaluacion}
        Instrumentos de evaluación
    \end{instrumentosEvaluacion}
\end{estrategiasEvaluacion}

\begin{requisitosAprobacion}
\item El alumno tendrá derecho a observar o en su defecto a ratificar las notas consignadas en sus evaluaciones, después de ser entregadas las mismas por parte del profesor, salvo el vencimiento de plazos para culminación del semestre académico, luego del mismo, no se admitirán reclamaciones,
alumno que no se haga presente en el día establecido, perderá su derecho a reclamo.
\item Para aprobar ...
\end{requisitosAprobacion}

\bibliography{CS341.bib}
\bibliographystyle{apalike}

\fecha
\firma

\end{document}


