\documentclass[a4paper,8pt]{article}
\usepackage[utf8]{inputenc}
\usepackage{tabularx}
\usepackage{setspace}
\usepackage{../../silabus}

\begin{document}

%%Setear variables
\setPeriodoAcademico{2018-B}
\setNombreAsignatura{Matemática aplicada a la computación}
\setNombreProfesor{}
\setGradoProfesorAbreviado{}
\sylabusHeader

\academicaTable
{Ciencia de la Computación} %Escuela Profesional
{MA1003234} %Código de la asignatura
{6$^{to}$ Semestre.} %Número del semestre
{Semestral} %Características
{17 Semanas} %Duración
{2 HT} %Número de horas teóricas
{2 HP} %Número de horas prácticas
{0} %Número de horas seminarios
{2 HL}  %Número de horas laboratorio
{2} %Número de horas Teórico-práctico
{4} %Número de créditos
{CB1003127} % Prerrequisitos (separados por comas)

\administrativaTable
{Doctor} %Grado académico del profesor
{Ingeniería de Sistemas e Informática} %Departamento académico
{6} %Número de horas totales
{2} %Número de horas - lunes
{-} %Número de horas - martes
{2} %Número de horas - miercoles
{2} %Número de horas - jueves
{-} %Número de horas - viernes
{101} %Aula de clase - lunes
{-} %Aula de clase - martes
{101} %Aula de clase - miercoles
{101} %Aula de clase - jueves
{-} %Aula de clase - viernes


\begin{fundamentacion}
Este curso es importante porque desarrolla tópicos del Álgebra Lineal y de Ecuaciones Diferenciales Ordinarias útiles en todas aquellas áreas de la ciencia de la computación donde se trabaja con sistemas lineales y sistemas dinámicos.

\end{fundamentacion}

\begin{sumilla}
\item Espacios Lineales
\item Transformaciones lineales
\item Autovalores y autovectores
\item Sistemas de ecuaciones diferenciales
\item Teoría fundamental
\item Estabilidad de equilibrio

\end{sumilla}

\begin{competenciasAsignatura}
\item \ShowCompetence{C1}{a}
\item \ShowCompetence{C20}{i}
\item \ShowCompetence{CS2}{j}

\end{competenciasAsignatura}

\begin{contenidos}


%inicio unidad
\nextUnidad{Espacios Lineales}
\nextCapitulo{Espacios Lineales}
\nextTema{Espacios vectoriales.}
\nextTema{Independencia, base y dimensión.}
\nextTema{Dimensiones y ortogonalidad de los cuatro subespacios.}
\nextTema{Aproximaciones por mínimos cuadrados.}
\nextTema{Proyecciones}
\nextTema{Bases ortogonales y Gram-Schmidt}

\cite{Strang03}, \cite{Apostol73} 


%inicio unidad
\nextUnidad{Transformaciones lineales}
\nextCapitulo{Transformaciones lineales}
\nextTema{Concepto de transformación lineal.}
\nextTema{Matriz de una transformación lineal.}
\nextTema{Cambio de base.}
\nextTema{Diagonalización y pseudoinversa}

\cite{Strang03}, \cite{Apostol73} 


%inicio unidad
\nextUnidad{Autovalores y autovectores}
\nextCapitulo{Autovalores y autovectores}
\nextTema{Diagonalización de una matriz}
\nextTema{Matrices simétricas}
\nextTema{Matrices definidas positivas}
\nextTema{Matrices similares}
\nextTema{La descomposición de valor singular}

\cite{Strang03}, \cite{Apostol73} 


%inicio unidad
\nextUnidad{Sistemas de ecuaciones diferenciales}
\nextCapitulo{Sistemas de ecuaciones diferenciales}
\nextTema{Exponencial de una matriz}
\nextTema{Teoremas de existencia y unicidad para sistemas lineales homogéneos con coeficientes constantes}
\nextTema{Sistemas lineales no homogéneas con coeficientes constantes.}

\cite{Zill02}, \cite{Apostol73} 


%inicio unidad
\nextUnidad{Teoría fundamental}
\nextCapitulo{Teoría fundamental}
\nextTema{Sistemas dinámicos}
\nextTema{El teorema fundamental}
\nextTema{Existencia y unicidad}
\nextTema{El flujo de una ecuación diferencial}

\cite{Hirsh74} 


%inicio unidad
\nextUnidad{Estabilidad de equilibrio}
\nextCapitulo{Estabilidad de equilibrio}
\nextTema{Estabilidad}
\nextTema{Funciones de Liapunov}
\nextTema{Sistemas gradientes}

\cite{Zill02}, \cite{Hirsh74} 





\end{contenidos}




\begin{estrategiasEnsenanza}
    \begin{metodos}
        Método expositivo en las clases teóricas \\
        Método de elaboración conjunta en los seminarios taller y en la elaboración del proyecto de investigación.
    \end{metodos}
    \begin{medios}
        Pizarra acrílica, plumones, cañón multimedia, material de laboratorio, videos, software.
    \end{medios}
    \begin{formasOrganizacion}
        %Se pone los que se necesiten
        \newItemFO{Clases Teóricas}{Desarrollo de los conceptos teóricos}
        \newItemFO{Seminarios}{Algo...}
        \newItemFO{Prácticas}{Algo...}
        \newItemFO{Laboratorio}{Aplicación de los conceptos vistos es clases teóricas.}
        \newItemFO{Otros}{Algo...}
    \end{formasOrganizacion}
    \begin{programacion}
        \newItemFO{Investigación Formativa}{Implementación de Sistema Computacional Web usando una base de datos relacional normalizada}
        \newItemFO{Responsabilidad Social}{Generar videos para la enseñanza de implementación de bases de datos y que sean disponibilizados de la población}
    \end{programacion}
    \begin{segumientoAprendizaje}
        Aquí va el seguimiento del aprendizaje
    \end{segumientoAprendizaje}
\end{estrategiasEnsenanza}


\begin{cronogramaAcademico}
    \newItemCA{Tema1}{Edward Hinojosa Cárdenas}{10}  %Tema/Evaluación - Docente - Porcentaje acumulado
    \newItemCA{Tema2}{Edward Hinojosa Cárdenas}{16}
    \newItemCA{Tema3}{Edward Hinojosa Cárdenas}{20}
    \newItemCA{Tema4}{Edward Hinojosa Cárdenas}{25}
    \newItemCA{Tema5}{Edward Hinojosa Cárdenas}{33}
    \newItemCA{Tema6}{Edward Hinojosa Cárdenas}{37}
    \newItemCA{Tema7}{Edward Hinojosa Cárdenas}{40}
    \newItemCA{Tema8}{Edward Hinojosa Cárdenas}{45}
    \newItemCA{Tema9}{Edward Hinojosa Cárdenas}{50}
    \newItemCA{Tema10}{Edward Hinojosa Cárdenas}{53}
    \newItemCA{Tema11}{Edward Hinojosa Cárdenas}{58}
    \newItemCA{Tema12}{Edward Hinojosa Cárdenas}{64}
    \newItemCA{Tema13}{Edward Hinojosa Cárdenas}{69}
    \newItemCA{Tema14}{Edward Hinojosa Cárdenas}{76}
    \newItemCA{Tema15}{Edward Hinojosa Cárdenas}{84}
    \newItemCA{Tema16}{Edward Hinojosa Cárdenas}{93}
    \newItemCA{Tema17}{Edward Hinojosa Cárdenas}{100}
\end{cronogramaAcademico}

\begin{estrategiasEvaluacion}
    \begin{evaluacionContinua}
        Práctica y Laboratorios en cada clase sobre los temas realizados, tanto para el primer parcial (EC1), segundo parcial (EC2) y tercer parcial (EC3).
    \end{evaluacionContinua}
    \begin{evaluacionPeriodica}
        \newItemEP{Primer Examen}{ponderación}
        \newItemEP{Segundo Examen}{ponderación}
        \newItemEP{Tercer Examen}{}
    \end{evaluacionPeriodica}
    \begin{cronogramaEvaluacion}
        \newItemCE{11/5/2019}{11/5/2019}{11/5/2019}{30\%}
        \newItemCE{11/8/2019}{11/8/2019}{11/8/2019}{30\%}
        \newItemCE{11/12/2019}{11/12/2019}{11/12/2019}{40\%}
    \end{cronogramaEvaluacion}
    \begin{tipoEvaluacion}
        Tipo de evaluación
    \end{tipoEvaluacion}
    \begin{instrumentosEvaluacion}
        Instrumentos de evaluación
    \end{instrumentosEvaluacion}
\end{estrategiasEvaluacion}

\begin{requisitosAprobacion}
\item El alumno tendrá derecho a observar o en su defecto a ratificar las notas consignadas en sus evaluaciones, después de ser entregadas las mismas por parte del profesor, salvo el vencimiento de plazos para culminación del semestre académico, luego del mismo, no se admitirán reclamaciones,
alumno que no se haga presente en el día establecido, perderá su derecho a reclamo.
\item Para aprobar ...
\end{requisitosAprobacion}

\bibliography{MA307.bib}
\bibliographystyle{apalike}

\fecha
\firma

\end{document}


