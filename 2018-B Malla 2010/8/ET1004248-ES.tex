\documentclass[a4paper,8pt]{article}
\usepackage[utf8]{inputenc}
\usepackage{tabularx}
\usepackage{setspace}
\usepackage{../../silabus}

\begin{document}

%%Setear variables
\setPeriodoAcademico{2018-B}
\setNombreAsignatura{Formación de Empresas de Base Tecnológica I}
\setNombreProfesor{}
\setGradoProfesorAbreviado{}
\sylabusHeader

\academicaTable
{Ciencia de la Computación} %Escuela Profesional
{ET1004248} %Código de la asignatura
{8$^{vo}$ Semestre.} %Número del semestre
{Semestral} %Características
{17 Semanas} %Duración
{2 HT} %Número de horas teóricas
{2 HP} %Número de horas prácticas
{0} %Número de horas seminarios
{}  %Número de horas laboratorio
{2} %Número de horas Teórico-práctico
{3} %Número de créditos
{CS1004137} % Prerrequisitos (separados por comas)

\administrativaTable
{Doctor} %Grado académico del profesor
{Ingeniería de Sistemas e Informática} %Departamento académico
{6} %Número de horas totales
{2} %Número de horas - lunes
{-} %Número de horas - martes
{2} %Número de horas - miercoles
{2} %Número de horas - jueves
{-} %Número de horas - viernes
{101} %Aula de clase - lunes
{-} %Aula de clase - martes
{101} %Aula de clase - miercoles
{101} %Aula de clase - jueves
{-} %Aula de clase - viernes


\begin{fundamentacion}
Este es el primer curso dentro del área de formación de empresas de
base tecnológica, tiene como objetivo dotar al futuro profesional 
de conocimientos, actitudes y aptitudes que le
permitan elaborar un plan de negocio para una empresa de base tecnológica.
El curso está dividido en las siguientes unidades:
Introducción, Creatividad, De la idea a la oportunidad, el modelo Canvas, Customer Development y Lean Startup, Aspectos Legales y Marketing, Finanzas de la empresa y Presentación.

Se busca aprovechar el potencial creativo e innovador y el esfuerzo de los alumnos en la creación de nuevas empresas.

\end{fundamentacion}

\begin{sumilla}
\item 
\item 
\item 
\item 
\item 
\item 
\item 
\item 

\end{sumilla}

\begin{competenciasAsignatura}
\item \ShowCompetence{C2}{d}
\item \ShowCompetence{C10}{f}
\item \ShowCompetence{C17}{f}
\item \ShowCompetence{C18}{i}
\item \ShowCompetence{C19}{i}
\item \ShowCompetence{C20}{k}
\item \ShowCompetence{C23}{k}
\item \ShowCompetence{CS5}{m}

\end{competenciasAsignatura}

\begin{contenidos}


%inicio unidad
\nextUnidad{}
\nextCapitulo{}
\nextTema{Emprendedor, emprendedurismo e innovación tecnológica}
\nextTema{Modelos de negocio}
\nextTema{Formación de equipos}

\cite{byers10}, \cite{osterwalder10}, \cite{garzozi14} 


%inicio unidad
\nextUnidad{}
\nextCapitulo{}
\nextTema{Visión}
\nextTema{Misión}
\nextTema{La Propuesta de valor}
\nextTema{Creatividad e invención}
\nextTema{Tipos y fuentes de innovación}
\nextTema{Estrategia y Tecnología}
\nextTema{Escala y ámbito}

\cite{byers10}, \cite{blank12}, \cite{garzozi14} 


%inicio unidad
\nextUnidad{}
\nextCapitulo{}
\nextTema{Estrategia de la Empresa}
\nextTema{Barreras}
\nextTema{Ventaja competitiva sostenible}
\nextTema{Alianzas}
\nextTema{Aprendizaje organizacional}
\nextTema{Desarrollo y diseño de productos}

\cite{byers10}, \cite{osterwalder10}, \cite{ries11}, \cite{garzozi14} 


%inicio unidad
\nextUnidad{}
\nextCapitulo{}
\nextTema{Creación de un nuevo negocio}
\nextTema{El plan de negocio}
\nextTema{Canvas}
\nextTema{Elementos del Canvas}

\cite{osterwalder10}, \cite{blank12}, \cite{garzozi14} 


%inicio unidad
\nextUnidad{}
\nextCapitulo{}
\nextTema{Aceleración versus incubación}
\nextTema{Customer Development}
\nextTema{Lean Startup}

\cite{blank12}, \cite{ries11}, \cite{garzozi14} 


%inicio unidad
\nextUnidad{}
\nextCapitulo{}
\nextTema{Aspectos Legales y tributarios para la constitución de la empresa}
\nextTema{Propiedad intelectual}
\nextTema{Patentes}
\nextTema{Copyrights y marca registrada}
\nextTema{Objetivos de marketing  y segmentos de mercado}
\nextTema{Investigación de mercado y búsqueda de clientes}

\cite{byers10}, \cite{ries11}, \cite{congreso96}, \cite{congreso97}, \cite{garzozi14} 


%inicio unidad
\nextUnidad{}
\nextCapitulo{}
\nextTema{Modelo de costos}
\nextTema{Modelo de utilidades}
\nextTema{Precio}
\nextTema{Plan financiero}
\nextTema{Formas de financiamiento}
\nextTema{Fuentes de capital}
\nextTema{Capital de riesgo}

\cite{byers10}, \cite{blank12}, \cite{garzozi14} 


%inicio unidad
\nextUnidad{}
\nextCapitulo{}
\nextTema{The Elevator Pitch}
\nextTema{Presentación}
\nextTema{Negociación}

\cite{byers10}, \cite{blank12}, \cite{garzozi14} 





\end{contenidos}




\begin{estrategiasEnsenanza}
    \begin{metodos}
        Método expositivo en las clases teóricas \\
        Método de elaboración conjunta en los seminarios taller y en la elaboración del proyecto de investigación.
    \end{metodos}
    \begin{medios}
        Pizarra acrílica, plumones, cañón multimedia, material de laboratorio, videos, software.
    \end{medios}
    \begin{formasOrganizacion}
        %Se pone los que se necesiten
        \newItemFO{Clases Teóricas}{Desarrollo de los conceptos teóricos}
        \newItemFO{Seminarios}{Algo...}
        \newItemFO{Prácticas}{Algo...}
        \newItemFO{Laboratorio}{Aplicación de los conceptos vistos es clases teóricas.}
        \newItemFO{Otros}{Algo...}
    \end{formasOrganizacion}
    \begin{programacion}
        \newItemFO{Investigación Formativa}{Implementación de Sistema Computacional Web usando una base de datos relacional normalizada}
        \newItemFO{Responsabilidad Social}{Generar videos para la enseñanza de implementación de bases de datos y que sean disponibilizados de la población}
    \end{programacion}
    \begin{segumientoAprendizaje}
        Aquí va el seguimiento del aprendizaje
    \end{segumientoAprendizaje}
\end{estrategiasEnsenanza}


\begin{cronogramaAcademico}
    \newItemCA{Tema1}{Edward Hinojosa Cárdenas}{10}  %Tema/Evaluación - Docente - Porcentaje acumulado
    \newItemCA{Tema2}{Edward Hinojosa Cárdenas}{16}
    \newItemCA{Tema3}{Edward Hinojosa Cárdenas}{20}
    \newItemCA{Tema4}{Edward Hinojosa Cárdenas}{25}
    \newItemCA{Tema5}{Edward Hinojosa Cárdenas}{33}
    \newItemCA{Tema6}{Edward Hinojosa Cárdenas}{37}
    \newItemCA{Tema7}{Edward Hinojosa Cárdenas}{40}
    \newItemCA{Tema8}{Edward Hinojosa Cárdenas}{45}
    \newItemCA{Tema9}{Edward Hinojosa Cárdenas}{50}
    \newItemCA{Tema10}{Edward Hinojosa Cárdenas}{53}
    \newItemCA{Tema11}{Edward Hinojosa Cárdenas}{58}
    \newItemCA{Tema12}{Edward Hinojosa Cárdenas}{64}
    \newItemCA{Tema13}{Edward Hinojosa Cárdenas}{69}
    \newItemCA{Tema14}{Edward Hinojosa Cárdenas}{76}
    \newItemCA{Tema15}{Edward Hinojosa Cárdenas}{84}
    \newItemCA{Tema16}{Edward Hinojosa Cárdenas}{93}
    \newItemCA{Tema17}{Edward Hinojosa Cárdenas}{100}
\end{cronogramaAcademico}

\begin{estrategiasEvaluacion}
    \begin{evaluacionContinua}
        Práctica y Laboratorios en cada clase sobre los temas realizados, tanto para el primer parcial (EC1), segundo parcial (EC2) y tercer parcial (EC3).
    \end{evaluacionContinua}
    \begin{evaluacionPeriodica}
        \newItemEP{Primer Examen}{ponderación}
        \newItemEP{Segundo Examen}{ponderación}
        \newItemEP{Tercer Examen}{}
    \end{evaluacionPeriodica}
    \begin{cronogramaEvaluacion}
        \newItemCE{11/5/2019}{11/5/2019}{11/5/2019}{30\%}
        \newItemCE{11/8/2019}{11/8/2019}{11/8/2019}{30\%}
        \newItemCE{11/12/2019}{11/12/2019}{11/12/2019}{40\%}
    \end{cronogramaEvaluacion}
    \begin{tipoEvaluacion}
        Tipo de evaluación
    \end{tipoEvaluacion}
    \begin{instrumentosEvaluacion}
        Instrumentos de evaluación
    \end{instrumentosEvaluacion}
\end{estrategiasEvaluacion}

\begin{requisitosAprobacion}
\item El alumno tendrá derecho a observar o en su defecto a ratificar las notas consignadas en sus evaluaciones, después de ser entregadas las mismas por parte del profesor, salvo el vencimiento de plazos para culminación del semestre académico, luego del mismo, no se admitirán reclamaciones,
alumno que no se haga presente en el día establecido, perderá su derecho a reclamo.
\item Para aprobar ...
\end{requisitosAprobacion}

\bibliography{ET201.bib}
\bibliographystyle{apalike}

\fecha
\firma

\end{document}


