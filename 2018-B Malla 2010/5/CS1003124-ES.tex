\documentclass[a4paper,8pt]{article}
\usepackage[utf8]{inputenc}
\usepackage{tabularx}
\usepackage{setspace}
\usepackage{../../silabus}

\begin{document}

%%Setear variables
\setPeriodoAcademico{2018-B}
\setNombreAsignatura{Análisis y Diseño de Algoritmos}
\setNombreProfesor{}
\setGradoProfesorAbreviado{}
\sylabusHeader

\academicaTable
{Ciencia de la Computación} %Escuela Profesional
{CS1003124} %Código de la asignatura
{5$^{to}$ Semestre.} %Número del semestre
{Semestral} %Características
{17 Semanas} %Duración
{2 HT} %Número de horas teóricas
{2 HP} %Número de horas prácticas
{0} %Número de horas seminarios
{2 HL}  %Número de horas laboratorio
{2} %Número de horas Teórico-práctico
{4} %Número de créditos
{CS1002221,MA1002223} % Prerrequisitos (separados por comas)

\administrativaTable
{Doctor} %Grado académico del profesor
{Ingeniería de Sistemas e Informática} %Departamento académico
{6} %Número de horas totales
{2} %Número de horas - lunes
{-} %Número de horas - martes
{2} %Número de horas - miercoles
{2} %Número de horas - jueves
{-} %Número de horas - viernes
{101} %Aula de clase - lunes
{-} %Aula de clase - martes
{101} %Aula de clase - miercoles
{101} %Aula de clase - jueves
{-} %Aula de clase - viernes


\begin{fundamentacion}
Un algoritmo es, esencialmente, un conjunto bien definido de reglas o instrucciones
que permitan resolver un problema computacional. El estudio teórico del desempeño
de los algoritmos y los recursos utilizados por estos, generalmente tiempo y espacio, 
nos permite evaluar si un algoritmo es adecuado para un resolver un problema 
específico, compararlo con otros algoritmos para el mismo problema o incluso
delimitar la frontera entre lo viable y lo imposible.

Esta materia es tan importante que incluso Donald E. Knuth definió a
Ciencia de la Computación como el estudio de algoritmos.

En este curso serán presentadas las técnicas más comunes utilizadas en el análisis y diseño de 
algoritmos eficientes, con el propósito de aprender los principios fundamentales
del diseño, implementación y análisis de algoritmos para la solución de problemas
computacionales.

\end{fundamentacion}

\begin{sumilla}
\item \ALBasicAnalysis
\item \ALAlgorithmicStrategies
\item \ALFundamentalDataStructuresandAlgorithms
\item \ALBasicAutomataComputabilityandComplexity
\item \ALAdvancedDataStructuresAlgorithmsandAnalysis

\end{sumilla}

\begin{competenciasAsignatura}
\item \ShowCompetence{C1}{a}
\item \ShowCompetence{C2}{b}
\item \ShowCompetence{C3}{b}
\item \ShowCompetence{C5}{a}
\item \ShowCompetence{C6}{a}
\item \ShowCompetence{C9}{a}

\end{competenciasAsignatura}

\begin{contenidos}


%inicio unidad
\nextUnidad{\ALBasicAnalysis}
\nextCapitulo{\ALBasicAnalysis}
\nextTema{\ALBasicAnalysisTopicDifferences % Diferencias entre el mejor, el esperado y el peor caso de un algoritmo.}
\nextTema{\ALBasicAnalysisTopicAsymptotic % Análisis asintótico de complejidad de cotas superior y esperada.}
\nextTema{\ALBasicAnalysisTopicBig % Definición formal de la Notación Big O.}
\nextTema{\ALBasicAnalysisTopicComplexity % Clases de complejidad como constante, logarítmica, lineal, cuadrática y exponencial.}
\nextTema{\ALBasicAnalysisTopicBigO % Uso de la notación Big O.}
\nextTema{\ALBasicAnalysisTopicRecurrence % Relaciones recurrentes.}
\nextTema{\ALBasicAnalysisTopicAnalysis % Análisis de algoritmos iterativos y recursivos.}
\nextTema{\ALBasicAnalysisTopicSome % Algunas versiones del Teorema Maestro.}

\cite{KT2005}, \cite{DPV2006}, \cite{CLRS2009}, \cite{S2013}, \cite{K1997} 


%inicio unidad
\nextUnidad{\ALAlgorithmicStrategies}
\nextCapitulo{\ALAlgorithmicStrategies}
\nextTema{\ALAlgorithmicStrategiesTopicBrute % Algoritmos de fuerza bruta.}
\nextTema{\ALAlgorithmicStrategiesTopicGreedy % Algoritmos voraces.}
\nextTema{\ALAlgorithmicStrategiesTopicDivide % Divide y vencerás.}
\nextTema{\ALAlgorithmicStrategiesTopicDynamic % Programación Dinámica.}

\cite{KT2005}, \cite{DPV2006}, \cite{CLRS2009}, \cite{A1999} 


%inicio unidad
\nextUnidad{\ALFundamentalDataStructuresandAlgorithms}
\nextCapitulo{\ALFundamentalDataStructuresandAlgorithms}
\nextTema{\ALFundamentalDataStructuresandAlgorithmsTopicSimple % Algoritmos numéricos simples, tales como el cálculo de la media de una lista de números, encontrar el mínimo y máximo.}
\nextTema{\ALFundamentalDataStructuresandAlgorithmsTopicSequential % Algoritmos de búsqueda secuencial y binaria.}
\nextTema{\ALFundamentalDataStructuresandAlgorithmsTopicWorst % Algoritmos de ordenamiento de peor caso cuadrático (selección, inserción)}
\nextTema{\ALFundamentalDataStructuresandAlgorithmsTopicWorstOr % Algoritmos de ordenamiento con peor caso o caso promedio en O(N lg N) (Quicksort, Heapsort, Mergesort)}
\nextTema{\ALFundamentalDataStructuresandAlgorithmsTopicGraphs % "Grafos y algoritmos en grafos:}
\nextTema{%  Representación de grafos (ej., lista de adyacencia, matriz de adyacencia)}
\nextTema{%  Recorrido en profundidad y amplitud}
\nextTema{\ALFundamentalDataStructuresandAlgorithmsTopicHeaps %Montículos (Heaps)}
\nextTema{\ALFundamentalDataStructuresandAlgorithmsTopicGraphsAnd % "Grafos y algoritmos en grafos:}
\nextTema{%  Algoritmos de la ruta más corta (algoritmos de Dijkstra y Floyd)}
\nextTema{%  Árbol de expansión mínima (algoritmos de Prim y Kruskal)}

\cite{KT2005}, \cite{DPV2006}, \cite{CLRS2009}, \cite{S2011}, \cite{GT2009} 


%inicio unidad
\nextUnidad{\ALBasicAutomataComputabilityandComplexity}
\nextCapitulo{\ALBasicAutomataComputabilityandComplexity}
\nextTema{\ALBasicAutomataComputabilityandComplexityTopicIntroduction % Introducción a las clases P y NP y al problema P vs. NP.}
\nextTema{\ALBasicAutomataComputabilityandComplexityTopicIntroductionTo % Introducción y ejemplos de problemas NP- Completos y a clases NP-Completos.}

\cite{KT2005}, \cite{DPV2006}, \cite{CLRS2009} 


%inicio unidad
\nextUnidad{\ALAdvancedDataStructuresAlgorithmsandAnalysis}
\nextCapitulo{\ALAdvancedDataStructuresAlgorithmsandAnalysis}
\nextTema{\ALAdvancedDataStructuresAlgorithmsandAnalysisTopicGraphs % Grafos (ej. Ordenamiento Topológico, encontrando componentes puertemente conectados)}
\nextTema{\ALAdvancedDataStructuresAlgorithmsandAnalysisTopicNumber % Algoritmos Teórico-Numéricos (Aritmética Modular, Prueba del Número Primo, Factorización Entera)}
\nextTema{\ALAdvancedDataStructuresAlgorithmsandAnalysisTopicRandomized % Algoritmos aleatorios.}
\nextTema{\ALAdvancedDataStructuresAlgorithmsandAnalysisTopicAmortized % Análisis amortizado.}
\nextTema{\ALAdvancedDataStructuresAlgorithmsandAnalysisTopicProbabilistic % Análisis Probabilístico.}

\cite{KT2005}, \cite{DPV2006}, \cite{CLRS2009}, \cite{T1983}, \cite{R1992} 





\end{contenidos}




\begin{estrategiasEnsenanza}
    \begin{metodos}
        Método expositivo en las clases teóricas \\
        Método de elaboración conjunta en los seminarios taller y en la elaboración del proyecto de investigación.
    \end{metodos}
    \begin{medios}
        Pizarra acrílica, plumones, cañón multimedia, material de laboratorio, videos, software.
    \end{medios}
    \begin{formasOrganizacion}
        %Se pone los que se necesiten
        \newItemFO{Clases Teóricas}{Desarrollo de los conceptos teóricos}
        \newItemFO{Seminarios}{Algo...}
        \newItemFO{Prácticas}{Algo...}
        \newItemFO{Laboratorio}{Aplicación de los conceptos vistos es clases teóricas.}
        \newItemFO{Otros}{Algo...}
    \end{formasOrganizacion}
    \begin{programacion}
        \newItemFO{Investigación Formativa}{Implementación de Sistema Computacional Web usando una base de datos relacional normalizada}
        \newItemFO{Responsabilidad Social}{Generar videos para la enseñanza de implementación de bases de datos y que sean disponibilizados de la población}
    \end{programacion}
    \begin{segumientoAprendizaje}
        Aquí va el seguimiento del aprendizaje
    \end{segumientoAprendizaje}
\end{estrategiasEnsenanza}


\begin{cronogramaAcademico}
    \newItemCA{Tema1}{Edward Hinojosa Cárdenas}{10}  %Tema/Evaluación - Docente - Porcentaje acumulado
    \newItemCA{Tema2}{Edward Hinojosa Cárdenas}{16}
    \newItemCA{Tema3}{Edward Hinojosa Cárdenas}{20}
    \newItemCA{Tema4}{Edward Hinojosa Cárdenas}{25}
    \newItemCA{Tema5}{Edward Hinojosa Cárdenas}{33}
    \newItemCA{Tema6}{Edward Hinojosa Cárdenas}{37}
    \newItemCA{Tema7}{Edward Hinojosa Cárdenas}{40}
    \newItemCA{Tema8}{Edward Hinojosa Cárdenas}{45}
    \newItemCA{Tema9}{Edward Hinojosa Cárdenas}{50}
    \newItemCA{Tema10}{Edward Hinojosa Cárdenas}{53}
    \newItemCA{Tema11}{Edward Hinojosa Cárdenas}{58}
    \newItemCA{Tema12}{Edward Hinojosa Cárdenas}{64}
    \newItemCA{Tema13}{Edward Hinojosa Cárdenas}{69}
    \newItemCA{Tema14}{Edward Hinojosa Cárdenas}{76}
    \newItemCA{Tema15}{Edward Hinojosa Cárdenas}{84}
    \newItemCA{Tema16}{Edward Hinojosa Cárdenas}{93}
    \newItemCA{Tema17}{Edward Hinojosa Cárdenas}{100}
\end{cronogramaAcademico}

\begin{estrategiasEvaluacion}
    \begin{evaluacionContinua}
        Práctica y Laboratorios en cada clase sobre los temas realizados, tanto para el primer parcial (EC1), segundo parcial (EC2) y tercer parcial (EC3).
    \end{evaluacionContinua}
    \begin{evaluacionPeriodica}
        \newItemEP{Primer Examen}{ponderación}
        \newItemEP{Segundo Examen}{ponderación}
        \newItemEP{Tercer Examen}{}
    \end{evaluacionPeriodica}
    \begin{cronogramaEvaluacion}
        \newItemCE{11/5/2019}{11/5/2019}{11/5/2019}{30\%}
        \newItemCE{11/8/2019}{11/8/2019}{11/8/2019}{30\%}
        \newItemCE{11/12/2019}{11/12/2019}{11/12/2019}{40\%}
    \end{cronogramaEvaluacion}
    \begin{tipoEvaluacion}
        Tipo de evaluación
    \end{tipoEvaluacion}
    \begin{instrumentosEvaluacion}
        Instrumentos de evaluación
    \end{instrumentosEvaluacion}
\end{estrategiasEvaluacion}

\begin{requisitosAprobacion}
\item El alumno tendrá derecho a observar o en su defecto a ratificar las notas consignadas en sus evaluaciones, después de ser entregadas las mismas por parte del profesor, salvo el vencimiento de plazos para culminación del semestre académico, luego del mismo, no se admitirán reclamaciones,
alumno que no se haga presente en el día establecido, perderá su derecho a reclamo.
\item Para aprobar ...
\end{requisitosAprobacion}

\bibliography{CS212.bib}
\bibliographystyle{apalike}

\fecha
\firma

\end{document}


