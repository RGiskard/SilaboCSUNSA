\documentclass[a4paper,8pt]{article}
\usepackage[utf8]{inputenc}
\usepackage{tabularx}
\usepackage{setspace}
\usepackage{../../silabus}

\begin{document}

%%Setear variables
\setPeriodoAcademico{2018-B}
\setNombreAsignatura{Física Computacional}
\setNombreProfesor{}
\setGradoProfesorAbreviado{}
\sylabusHeader

\academicaTable
{Ciencia de la Computación} %Escuela Profesional
{CB1003127} %Código de la asignatura
{5$^{to}$ Semestre.} %Número del semestre
{Semestral} %Características
{17 Semanas} %Duración
{2 HT} %Número de horas teóricas
{2 HP} %Número de horas prácticas
{0} %Número de horas seminarios
{2 HL}  %Número de horas laboratorio
{2} %Número de horas Teórico-práctico
{4} %Número de créditos
{MA1002117} % Prerrequisitos (separados por comas)

\administrativaTable
{Doctor} %Grado académico del profesor
{Ingeniería de Sistemas e Informática} %Departamento académico
{6} %Número de horas totales
{2} %Número de horas - lunes
{-} %Número de horas - martes
{2} %Número de horas - miercoles
{2} %Número de horas - jueves
{-} %Número de horas - viernes
{101} %Aula de clase - lunes
{-} %Aula de clase - martes
{101} %Aula de clase - miercoles
{101} %Aula de clase - jueves
{-} %Aula de clase - viernes


\begin{fundamentacion}
Física I es un curso que le permitirá al estudiante entender
las leyes de física de macropartículas y micropartículas considerado desde un
punto material hasta un sistemas de partículas; debiéndose tener en cuenta que los
fenómenos aquí estudiados se relacionan a la física clásica: Cinemática, Dinámica, Trabajo y Energía; 
además se debe asociar que éstos problemas deben ser resueltos con algoritmos computacionales.

Poseer capacidad y habilidad en la interpretación de problemas clásicos
con condiciones de frontera reales que contribuyen en la elaboración de soluciones eficientes
y factibles en diferentes áreas de la Ciencia de la Computación.

\end{fundamentacion}

\begin{sumilla}
\item 
\item 
\item 
\item 
\item 
\item 

\end{sumilla}

\begin{competenciasAsignatura}
\item \ShowCompetence{C1}{a}
\item \ShowCompetence{C20}{i}
\item \ShowCompetence{CS2}{j}

\end{competenciasAsignatura}

\begin{contenidos}


%inicio unidad
\nextUnidad{}
\nextCapitulo{}
\nextTema{Análisis dimensional.}
\nextTema{Vectores. Propiedades. Operaciones.}
\nextTema{Caso práctico: Estimación de fuerzas.}

\cite{Burbano}, \cite{ResnikHalliday}, \cite{SerwayJewett}, \cite{TriplerMosca} 


%inicio unidad
\nextUnidad{}
\nextCapitulo{}
\nextTema{Primera y tercera Ley de Newton.}
\nextTema{Diagrama de cuerpo libre.}
\nextTema{Primera condición de equilibrio.}
\nextTema{Caso práctico: Estimación de la fuerza humana.}
\nextTema{Segunda condición de equilibrio.}
\nextTema{Torque.}
\nextTema{Casos prácticos: Aplicaciones en dispositivos mecánicos.}
\nextTema{Fricción.}

\cite{Burbano}, \cite{ResnikHalliday}, \cite{SerwayJewett}, \cite{TriplerMosca} 


%inicio unidad
\nextUnidad{}
\nextCapitulo{}
\nextTema{Posición, Velocidad, Aceleración.}
\nextTema{Gráficas de movimiento.}
\nextTema{Casos prácticos: Representación gráfica de movimiento utilizando Excel.}
\nextTema{Movimiento circular.}
\nextTema{Velocidad angular y velocidad tangencial.}
\nextTema{Mecanismos rotativos.}
\nextTema{Caso práctico: Operación de la caja de cambios de un automóvil.}

\cite{Burbano}, \cite{ResnikHalliday}, \cite{SerwayJewett}, \cite{TriplerMosca} 


%inicio unidad
\nextUnidad{}
\nextCapitulo{}
\nextTema{Segunda Ley de Newton.}
\nextTema{Fuerza y movimiento.}
\nextTema{Momento de inercia.}

\cite{Burbano}, \cite{ResnikHalliday}, \cite{SerwayJewett}, \cite{TriplerMosca} 


%inicio unidad
\nextUnidad{}
\nextCapitulo{}
\nextTema{Trabajo.}
\nextTema{Fuerzas constantes.}
\nextTema{Fuerzas variables.}
\nextTema{Potencia.}
\nextTema{Caso práctico: Estimación de la potencia de una planta hidroeléctrica.}

\cite{Burbano}, \cite{ResnikHalliday}, \cite{SerwayJewett}, \cite{TriplerMosca} 


%inicio unidad
\nextUnidad{}
\nextCapitulo{}
\nextTema{Tipos de energía.}
\nextTema{Conservación de la energía.}
\nextTema{Dinámica de un sistema de partículas.}
\nextTema{Colisiones.}

\cite{Burbano}, \cite{ResnikHalliday}, \cite{SerwayJewett}, \cite{TriplerMosca} 





\end{contenidos}




\begin{estrategiasEnsenanza}
    \begin{metodos}
        Método expositivo en las clases teóricas \\
        Método de elaboración conjunta en los seminarios taller y en la elaboración del proyecto de investigación.
    \end{metodos}
    \begin{medios}
        Pizarra acrílica, plumones, cañón multimedia, material de laboratorio, videos, software.
    \end{medios}
    \begin{formasOrganizacion}
        %Se pone los que se necesiten
        \newItemFO{Clases Teóricas}{Desarrollo de los conceptos teóricos}
        \newItemFO{Seminarios}{Algo...}
        \newItemFO{Prácticas}{Algo...}
        \newItemFO{Laboratorio}{Aplicación de los conceptos vistos es clases teóricas.}
        \newItemFO{Otros}{Algo...}
    \end{formasOrganizacion}
    \begin{programacion}
        \newItemFO{Investigación Formativa}{Implementación de Sistema Computacional Web usando una base de datos relacional normalizada}
        \newItemFO{Responsabilidad Social}{Generar videos para la enseñanza de implementación de bases de datos y que sean disponibilizados de la población}
    \end{programacion}
    \begin{segumientoAprendizaje}
        Aquí va el seguimiento del aprendizaje
    \end{segumientoAprendizaje}
\end{estrategiasEnsenanza}


\begin{cronogramaAcademico}
    \newItemCA{Tema1}{Edward Hinojosa Cárdenas}{10}  %Tema/Evaluación - Docente - Porcentaje acumulado
    \newItemCA{Tema2}{Edward Hinojosa Cárdenas}{16}
    \newItemCA{Tema3}{Edward Hinojosa Cárdenas}{20}
    \newItemCA{Tema4}{Edward Hinojosa Cárdenas}{25}
    \newItemCA{Tema5}{Edward Hinojosa Cárdenas}{33}
    \newItemCA{Tema6}{Edward Hinojosa Cárdenas}{37}
    \newItemCA{Tema7}{Edward Hinojosa Cárdenas}{40}
    \newItemCA{Tema8}{Edward Hinojosa Cárdenas}{45}
    \newItemCA{Tema9}{Edward Hinojosa Cárdenas}{50}
    \newItemCA{Tema10}{Edward Hinojosa Cárdenas}{53}
    \newItemCA{Tema11}{Edward Hinojosa Cárdenas}{58}
    \newItemCA{Tema12}{Edward Hinojosa Cárdenas}{64}
    \newItemCA{Tema13}{Edward Hinojosa Cárdenas}{69}
    \newItemCA{Tema14}{Edward Hinojosa Cárdenas}{76}
    \newItemCA{Tema15}{Edward Hinojosa Cárdenas}{84}
    \newItemCA{Tema16}{Edward Hinojosa Cárdenas}{93}
    \newItemCA{Tema17}{Edward Hinojosa Cárdenas}{100}
\end{cronogramaAcademico}

\begin{estrategiasEvaluacion}
    \begin{evaluacionContinua}
        Práctica y Laboratorios en cada clase sobre los temas realizados, tanto para el primer parcial (EC1), segundo parcial (EC2) y tercer parcial (EC3).
    \end{evaluacionContinua}
    \begin{evaluacionPeriodica}
        \newItemEP{Primer Examen}{ponderación}
        \newItemEP{Segundo Examen}{ponderación}
        \newItemEP{Tercer Examen}{}
    \end{evaluacionPeriodica}
    \begin{cronogramaEvaluacion}
        \newItemCE{11/5/2019}{11/5/2019}{11/5/2019}{30\%}
        \newItemCE{11/8/2019}{11/8/2019}{11/8/2019}{30\%}
        \newItemCE{11/12/2019}{11/12/2019}{11/12/2019}{40\%}
    \end{cronogramaEvaluacion}
    \begin{tipoEvaluacion}
        Tipo de evaluación
    \end{tipoEvaluacion}
    \begin{instrumentosEvaluacion}
        Instrumentos de evaluación
    \end{instrumentosEvaluacion}
\end{estrategiasEvaluacion}

\begin{requisitosAprobacion}
\item El alumno tendrá derecho a observar o en su defecto a ratificar las notas consignadas en sus evaluaciones, después de ser entregadas las mismas por parte del profesor, salvo el vencimiento de plazos para culminación del semestre académico, luego del mismo, no se admitirán reclamaciones,
alumno que no se haga presente en el día establecido, perderá su derecho a reclamo.
\item Para aprobar ...
\end{requisitosAprobacion}

\bibliography{CB111.bib}
\bibliographystyle{apalike}

\fecha
\firma

\end{document}


