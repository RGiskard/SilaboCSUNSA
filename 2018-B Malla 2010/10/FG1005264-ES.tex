\documentclass[a4paper,8pt]{article}
\usepackage[utf8]{inputenc}
\usepackage{tabularx}
\usepackage{setspace}
\usepackage{../../silabus}

\begin{document}

%%Setear variables
\setPeriodoAcademico{2018-B}
\setNombreAsignatura{Ética Profesional}
\setNombreProfesor{}
\setGradoProfesorAbreviado{}
\sylabusHeader

\academicaTable
{Ciencia de la Computación} %Escuela Profesional
{FG1005264} %Código de la asignatura
{10$^{mo}$ Semestre.} %Número del semestre
{Semestral} %Características
{17 Semanas} %Duración
{2 HT} %Número de horas teóricas
{2 HP} %Número de horas prácticas
{0} %Número de horas seminarios
{}  %Número de horas laboratorio
{2} %Número de horas Teórico-práctico
{3} %Número de créditos
{CS1004242} % Prerrequisitos (separados por comas)

\administrativaTable
{Doctor} %Grado académico del profesor
{Ingeniería de Sistemas e Informática} %Departamento académico
{6} %Número de horas totales
{2} %Número de horas - lunes
{-} %Número de horas - martes
{2} %Número de horas - miercoles
{2} %Número de horas - jueves
{-} %Número de horas - viernes
{101} %Aula de clase - lunes
{-} %Aula de clase - martes
{101} %Aula de clase - miercoles
{101} %Aula de clase - jueves
{-} %Aula de clase - viernes


\begin{fundamentacion}
La ética es una parte constitutiva inherente al ser humano, y como tal debe plasmarse en el actuar cotidiano y profesional de la persona humana. Es indispensable que la persona asuma su rol activo en la sociedad pues los sistemas económico-industrial, político y social no siempre están en función de valores y principios, siendo éstos en realidad los pilares sobre los que debería basarse todo el actuar de los profesionales.

\end{fundamentacion}

\begin{sumilla}
\item 
\item 
\item 
\item 

\end{sumilla}

\begin{competenciasAsignatura}
\item \ShowCompetence{C10}{n,ñ,o}
\item \ShowCompetence{C20}{n,ñ,o}
\item \ShowCompetence{C21}{n,ñ,o}
\item \ShowCompetence{C22}{n,ñ,o}

\end{competenciasAsignatura}

\begin{contenidos}


%inicio unidad
\nextUnidad{}
\nextCapitulo{}
\nextTema{Ser profesional y ser moral.}
\nextTema{La objetividad moral y la formulación de principios morales.}
\nextTema{El profesional y sus valores.}
\nextTema{La conciencia moral de la persona.}
\nextTema{El aporte de la DSI en el quehacer profesional.}
\nextTema{El bien común y el principio de subsidiaridad.}
\nextTema{Principios morales y propiedad privada.}
\nextTema{Justicia: Algunos conceptos básicos.}

\cite{ACM1992}, \cite{Schmidt1995}, \cite{Loza2000}, \cite{Argandon2006} 


%inicio unidad
\nextUnidad{}
\nextCapitulo{}
\nextTema{La responsabilidad individual del trabajador en la empresa.}
\nextTema{Liderazgo y ética profesional en el entorno laboral.}
\nextTema{Principios generales sobre la colaboración en hechos inmorales.}
\nextTema{El profesional frente al soborno: ?`víctima o colaboración?}

\cite{ACM1992}, \cite{Manzone2007}, \cite{Schmidt1995}, \cite{Perez1998}, \cite{Nieburh2003} 


%inicio unidad
\nextUnidad{}
\nextCapitulo{}
\nextTema{La ética profesional frente a la ética general.}
\nextTema{Trabajo y profesión en los tiempos actuales.}
\nextTema{Ética, ciencia y tecnología.}
\nextTema{Valores éticos en organizaciones relacionadas con el uso de la información.}
\nextTema{Valores éticos en la era de la Sociedad de la Información.}

\cite{ACM1992}, \cite{IEEE2004}, \cite{Hernandez2006} 


%inicio unidad
\nextUnidad{}
\nextCapitulo{}
\nextTema{Ética informática.}
\nextTema{Ética y software.}
\nextTema{El software libre.}
\nextTema{Regulación y ética de telecomunicaciones.}
\nextTema{Ética en Internet.}
\nextTema{Derechos de autor y patentes.}
\nextTema{Ética en los servicios de consultoría.}
\nextTema{Ética en los procesos de innovación tecnológica.}
\nextTema{Ética en la gestión tecnológica y en empresas de base tecnológica.}

\cite{Comunicaciones2002}, \cite{Hernandez2006}, \cite{ACM1992} 





\end{contenidos}




\begin{estrategiasEnsenanza}
    \begin{metodos}
        Método expositivo en las clases teóricas \\
        Método de elaboración conjunta en los seminarios taller y en la elaboración del proyecto de investigación.
    \end{metodos}
    \begin{medios}
        Pizarra acrílica, plumones, cañón multimedia, material de laboratorio, videos, software.
    \end{medios}
    \begin{formasOrganizacion}
        %Se pone los que se necesiten
        \newItemFO{Clases Teóricas}{Desarrollo de los conceptos teóricos}
        \newItemFO{Seminarios}{Algo...}
        \newItemFO{Prácticas}{Algo...}
        \newItemFO{Laboratorio}{Aplicación de los conceptos vistos es clases teóricas.}
        \newItemFO{Otros}{Algo...}
    \end{formasOrganizacion}
    \begin{programacion}
        \newItemFO{Investigación Formativa}{Implementación de Sistema Computacional Web usando una base de datos relacional normalizada}
        \newItemFO{Responsabilidad Social}{Generar videos para la enseñanza de implementación de bases de datos y que sean disponibilizados de la población}
    \end{programacion}
    \begin{segumientoAprendizaje}
        Aquí va el seguimiento del aprendizaje
    \end{segumientoAprendizaje}
\end{estrategiasEnsenanza}


\begin{cronogramaAcademico}
    \newItemCA{Tema1}{Edward Hinojosa Cárdenas}{10}  %Tema/Evaluación - Docente - Porcentaje acumulado
    \newItemCA{Tema2}{Edward Hinojosa Cárdenas}{16}
    \newItemCA{Tema3}{Edward Hinojosa Cárdenas}{20}
    \newItemCA{Tema4}{Edward Hinojosa Cárdenas}{25}
    \newItemCA{Tema5}{Edward Hinojosa Cárdenas}{33}
    \newItemCA{Tema6}{Edward Hinojosa Cárdenas}{37}
    \newItemCA{Tema7}{Edward Hinojosa Cárdenas}{40}
    \newItemCA{Tema8}{Edward Hinojosa Cárdenas}{45}
    \newItemCA{Tema9}{Edward Hinojosa Cárdenas}{50}
    \newItemCA{Tema10}{Edward Hinojosa Cárdenas}{53}
    \newItemCA{Tema11}{Edward Hinojosa Cárdenas}{58}
    \newItemCA{Tema12}{Edward Hinojosa Cárdenas}{64}
    \newItemCA{Tema13}{Edward Hinojosa Cárdenas}{69}
    \newItemCA{Tema14}{Edward Hinojosa Cárdenas}{76}
    \newItemCA{Tema15}{Edward Hinojosa Cárdenas}{84}
    \newItemCA{Tema16}{Edward Hinojosa Cárdenas}{93}
    \newItemCA{Tema17}{Edward Hinojosa Cárdenas}{100}
\end{cronogramaAcademico}

\begin{estrategiasEvaluacion}
    \begin{evaluacionContinua}
        Práctica y Laboratorios en cada clase sobre los temas realizados, tanto para el primer parcial (EC1), segundo parcial (EC2) y tercer parcial (EC3).
    \end{evaluacionContinua}
    \begin{evaluacionPeriodica}
        \newItemEP{Primer Examen}{ponderación}
        \newItemEP{Segundo Examen}{ponderación}
        \newItemEP{Tercer Examen}{}
    \end{evaluacionPeriodica}
    \begin{cronogramaEvaluacion}
        \newItemCE{11/5/2019}{11/5/2019}{11/5/2019}{30\%}
        \newItemCE{11/8/2019}{11/8/2019}{11/8/2019}{30\%}
        \newItemCE{11/12/2019}{11/12/2019}{11/12/2019}{40\%}
    \end{cronogramaEvaluacion}
    \begin{tipoEvaluacion}
        Tipo de evaluación
    \end{tipoEvaluacion}
    \begin{instrumentosEvaluacion}
        Instrumentos de evaluación
    \end{instrumentosEvaluacion}
\end{estrategiasEvaluacion}

\begin{requisitosAprobacion}
\item El alumno tendrá derecho a observar o en su defecto a ratificar las notas consignadas en sus evaluaciones, después de ser entregadas las mismas por parte del profesor, salvo el vencimiento de plazos para culminación del semestre académico, luego del mismo, no se admitirán reclamaciones,
alumno que no se haga presente en el día establecido, perderá su derecho a reclamo.
\item Para aprobar ...
\end{requisitosAprobacion}

\bibliography{FG211.bib}
\bibliographystyle{apalike}

\fecha
\firma

\end{document}


