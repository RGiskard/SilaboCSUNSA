\documentclass[a4paper,8pt]{article}
\usepackage[utf8]{inputenc}
\usepackage{tabularx}
\usepackage{setspace}
\usepackage{../../silabus}

\begin{document}

%%Setear variables
\setPeriodoAcademico{2018-B}
\setNombreAsignatura{Metodología del Estudio}
\setNombreProfesor{}
\setGradoProfesorAbreviado{}
\sylabusHeader

\academicaTable
{Ciencia de la Computación} %Escuela Profesional
{FG1001104} %Código de la asignatura
{1$^{er}$ Semestre.} %Número del semestre
{Semestral} %Características
{17 Semanas} %Duración
{} %Número de horas teóricas
{4 HP} %Número de horas prácticas
{0} %Número de horas seminarios
{}  %Número de horas laboratorio
{2} %Número de horas Teórico-práctico
{2} %Número de créditos
{Ninguno} % Prerrequisitos (separados por comas)

\administrativaTable
{Doctor} %Grado académico del profesor
{Ingeniería de Sistemas e Informática} %Departamento académico
{6} %Número de horas totales
{2} %Número de horas - lunes
{-} %Número de horas - martes
{2} %Número de horas - miercoles
{2} %Número de horas - jueves
{-} %Número de horas - viernes
{101} %Aula de clase - lunes
{-} %Aula de clase - martes
{101} %Aula de clase - miercoles
{101} %Aula de clase - jueves
{-} %Aula de clase - viernes


\begin{fundamentacion}
Los alumnos en formación profesional necesitan mejorar su actitud frente al trabajo y exigencia académicos. Además conviene que entiendan el proceso mental que se da en el ejercicio del estudio para lograr el aprendizaje; así  sabrán dónde y cómo hacer los ajustes más convenientes a sus necesidades. Asimismo, requieren dominar variadas formas de estudiar, para que puedan seleccionar las estrategias  más convenientes a su personal estilo de aprender y a la naturaleza de cada asignatura. De igual modo conocer y usar  maneras de buscar información académica y realizar trabajos creativos de tipo académico formal, así podrán  aplicarlos a su trabajo universitario, haciendo exitoso su esfuerzo.

\end{fundamentacion}

\begin{sumilla}
\item 
\item 
\item 
\item 

\end{sumilla}

\begin{competenciasAsignatura}
\item \ShowCompetence{C19}{h}
\item \ShowCompetence{C24}{h,d}

\end{competenciasAsignatura}

\begin{contenidos}


%inicio unidad
\nextUnidad{}
\nextCapitulo{}
\nextTema{El subrayado.}
\nextTema{Toma de puntes.}
\nextTema{La vocación, hábitos de la vida universitaria.}
\nextTema{Interacción humana.}
\nextTema{La voluntad como requisito para el aprendizaje.}
\nextTema{La plantificación y el tiempo}

\cite{Bibliografía} 


%inicio unidad
\nextUnidad{}
\nextCapitulo{}
\nextTema{Resumen. Notas al margen. Nemotecnias.}
\nextTema{Procesos mentales: Simples, complejos. Fundamentos del aprendizaje significativo.}
\nextTema{Los pasos o factores para el aprendizaje. Leyes del aprendizaje. Cuestionario de estilos de aprendizaje Identificación del estilo de aprendizaje personal}
\nextTema{La lectura académica. Niveles de  análisis de un texto: idea central, idea principal e ideas secundarias. El modelo de Meza de Vernet.}
\nextTema{Exámenes: Preparación. Pautas y estrategias para antes, durante y después de un examen. Inteligencia emocional y exámenes.}
\nextTema{Las fuentes de información. Aparato crítico: concepto y finalidad. Normas Vancouver. Referencias y citas.}

\cite{Rodriguez}, \cite{Pereza}, \cite{Quintana} 


%inicio unidad
\nextUnidad{}
\nextCapitulo{}
\nextTema{Los mapas conceptuales. Características y elementos.}
\nextTema{Los derechos de autor y el plagio. Derechos personales o morales. Derechos patrimoniales. ``Copyrigth''.}
\nextTema{Autoestima, Inteligencia Emocional, Asertividad y Resiliencia. Conceptos, desarrollo y fortalecimiento.}
\nextTema{Aparato crítico: Normas Vancouver. Aplicación práctica.}
\nextTema{Generación de ideas. Estrategias para organizar las ideas, redacción y revisión.}

\cite{Chaveza}, \cite{Flores} 


%inicio unidad
\nextUnidad{}
\nextCapitulo{}
\nextTema{Cuadro Sinóptico. Los mapas mentales. Practicas con la temática del curso.}
\nextTema{El método personal de estudio.}
\nextTema{El aprendizaje cooperativo: definición, los grupos de estudio, organización, roles de los miembros.}
\nextTema{Pautas para conformar grupos eficientes y armónicos.}
\nextTema{El método personal de estudio.Reforzamiento de técnicas de estudio.}
\nextTema{Presentación y exposición de trabajos de producción intelectual.}
\nextTema{El debate y la argumentación.}

\cite{Rodriguez}, \cite{Chaveza} 





\end{contenidos}




\begin{estrategiasEnsenanza}
    \begin{metodos}
        Método expositivo en las clases teóricas \\
        Método de elaboración conjunta en los seminarios taller y en la elaboración del proyecto de investigación.
    \end{metodos}
    \begin{medios}
        Pizarra acrílica, plumones, cañón multimedia, material de laboratorio, videos, software.
    \end{medios}
    \begin{formasOrganizacion}
        %Se pone los que se necesiten
        \newItemFO{Clases Teóricas}{Desarrollo de los conceptos teóricos}
        \newItemFO{Seminarios}{Algo...}
        \newItemFO{Prácticas}{Algo...}
        \newItemFO{Laboratorio}{Aplicación de los conceptos vistos es clases teóricas.}
        \newItemFO{Otros}{Algo...}
    \end{formasOrganizacion}
    \begin{programacion}
        \newItemFO{Investigación Formativa}{Implementación de Sistema Computacional Web usando una base de datos relacional normalizada}
        \newItemFO{Responsabilidad Social}{Generar videos para la enseñanza de implementación de bases de datos y que sean disponibilizados de la población}
    \end{programacion}
    \begin{segumientoAprendizaje}
        Aquí va el seguimiento del aprendizaje
    \end{segumientoAprendizaje}
\end{estrategiasEnsenanza}


\begin{cronogramaAcademico}
    \newItemCA{Tema1}{Edward Hinojosa Cárdenas}{10}  %Tema/Evaluación - Docente - Porcentaje acumulado
    \newItemCA{Tema2}{Edward Hinojosa Cárdenas}{16}
    \newItemCA{Tema3}{Edward Hinojosa Cárdenas}{20}
    \newItemCA{Tema4}{Edward Hinojosa Cárdenas}{25}
    \newItemCA{Tema5}{Edward Hinojosa Cárdenas}{33}
    \newItemCA{Tema6}{Edward Hinojosa Cárdenas}{37}
    \newItemCA{Tema7}{Edward Hinojosa Cárdenas}{40}
    \newItemCA{Tema8}{Edward Hinojosa Cárdenas}{45}
    \newItemCA{Tema9}{Edward Hinojosa Cárdenas}{50}
    \newItemCA{Tema10}{Edward Hinojosa Cárdenas}{53}
    \newItemCA{Tema11}{Edward Hinojosa Cárdenas}{58}
    \newItemCA{Tema12}{Edward Hinojosa Cárdenas}{64}
    \newItemCA{Tema13}{Edward Hinojosa Cárdenas}{69}
    \newItemCA{Tema14}{Edward Hinojosa Cárdenas}{76}
    \newItemCA{Tema15}{Edward Hinojosa Cárdenas}{84}
    \newItemCA{Tema16}{Edward Hinojosa Cárdenas}{93}
    \newItemCA{Tema17}{Edward Hinojosa Cárdenas}{100}
\end{cronogramaAcademico}

\begin{estrategiasEvaluacion}
    \begin{evaluacionContinua}
        Práctica y Laboratorios en cada clase sobre los temas realizados, tanto para el primer parcial (EC1), segundo parcial (EC2) y tercer parcial (EC3).
    \end{evaluacionContinua}
    \begin{evaluacionPeriodica}
        \newItemEP{Primer Examen}{ponderación}
        \newItemEP{Segundo Examen}{ponderación}
        \newItemEP{Tercer Examen}{}
    \end{evaluacionPeriodica}
    \begin{cronogramaEvaluacion}
        \newItemCE{11/5/2019}{11/5/2019}{11/5/2019}{30\%}
        \newItemCE{11/8/2019}{11/8/2019}{11/8/2019}{30\%}
        \newItemCE{11/12/2019}{11/12/2019}{11/12/2019}{40\%}
    \end{cronogramaEvaluacion}
    \begin{tipoEvaluacion}
        Tipo de evaluación
    \end{tipoEvaluacion}
    \begin{instrumentosEvaluacion}
        Instrumentos de evaluación
    \end{instrumentosEvaluacion}
\end{estrategiasEvaluacion}

\begin{requisitosAprobacion}
\item El alumno tendrá derecho a observar o en su defecto a ratificar las notas consignadas en sus evaluaciones, después de ser entregadas las mismas por parte del profesor, salvo el vencimiento de plazos para culminación del semestre académico, luego del mismo, no se admitirán reclamaciones,
alumno que no se haga presente en el día establecido, perderá su derecho a reclamo.
\item Para aprobar ...
\end{requisitosAprobacion}

\bibliography{FG101.bib}
\bibliographystyle{apalike}

\fecha
\firma

\end{document}


