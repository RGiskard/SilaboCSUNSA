\documentclass[a4paper,8pt]{article}
\usepackage[utf8]{inputenc}
\usepackage{tabularx}
\usepackage{setspace}
\usepackage{../../silabus}

\begin{document}

%%Setear variables
\setPeriodoAcademico{2018-B}
\setNombreAsignatura{Big Data}
\setNombreProfesor{}
\setGradoProfesorAbreviado{}
\sylabusHeader

\academicaTable
{Ciencia de la Computación} %Escuela Profesional
{CS1705157} %Código de la asignatura
{9$^{no}$ Semestre.} %Número del semestre
{Semestral} %Características
{17 Semanas} %Duración
{1 HT} %Número de horas teóricas
{2 HP} %Número de horas prácticas
{0} %Número de horas seminarios
{2 HL}  %Número de horas laboratorio
{2} %Número de horas Teórico-práctico
{3} %Número de créditos
{CS1703130,CS1704250} % Prerrequisitos (separados por comas)

\administrativaTable
{Doctor} %Grado académico del profesor
{Ingeniería de Sistemas e Informática} %Departamento académico
{6} %Número de horas totales
{2} %Número de horas - lunes
{-} %Número de horas - martes
{2} %Número de horas - miercoles
{2} %Número de horas - jueves
{-} %Número de horas - viernes
{101} %Aula de clase - lunes
{-} %Aula de clase - martes
{101} %Aula de clase - miercoles
{101} %Aula de clase - jueves
{-} %Aula de clase - viernes


\begin{fundamentacion}
En la actualidad conocer enfoques escalables para procesar y almacenar grande volumenes de información (terabytes, petabytes e inclusive exabytes) es fundamental en cursos de ciencia de la computación. Cada dia, cada hora, cada minuto se genera gran cantidad de información la cual necesitá ser procesada, almacenada, analisada.

\end{fundamentacion}

\begin{sumilla}
\item Introducción a Big Data
\item Hadoop
\item Procesamiento de Grafos en larga escala

\end{sumilla}

\begin{competenciasAsignatura}
\item \ShowCompetence{C2}{a,b}
\item \ShowCompetence{C16}{i}
\item \ShowCompetence{CS2}{i,b}
\item \ShowCompetence{CS3}{j}
\item \ShowCompetence{CS6}{j}

\end{competenciasAsignatura}

\begin{contenidos}


%inicio unidad
\nextUnidad{Introducción a Big Data}
\nextCapitulo{Introducción a Big Data}
\nextTema{Visión global sobre Cloud Computing}
\nextTema{Visión global sobre Sistema de Archivos Distribuidos%}
\nextTema{Visión global sobre el modelo de programación MapReduce%}

\cite{coulouris} 


%inicio unidad
\nextUnidad{Hadoop}
\nextCapitulo{Hadoop}
\nextTema{Visión global de Hadoop.}
\nextTema{Historia.}
\nextTema{Estructura de Hadoop.}
\nextTema{HDFS, Hadoop Distributed File System.}
\nextTema{Modelo de Programación MapReduce}

\cite{dongarra}, \cite{buyya} 


%inicio unidad
\nextUnidad{Procesamiento de Grafos en larga escala}
\nextCapitulo{Procesamiento de Grafos en larga escala}
\nextTema{Pregel: A System for Large-scale Graph Processing.}
\nextTema{Distributed GraphLab: A Framework for Machine Learning and Data Mining in the Cloud.}
\nextTema{Apache Giraph is an iterative graph processing system built for high scalability.}

\cite{graphlab}, \cite{pregel}, \cite{giraph} 





\end{contenidos}




\begin{estrategiasEnsenanza}
    \begin{metodos}
        Método expositivo en las clases teóricas \\
        Método de elaboración conjunta en los seminarios taller y en la elaboración del proyecto de investigación.
    \end{metodos}
    \begin{medios}
        Pizarra acrílica, plumones, cañón multimedia, material de laboratorio, videos, software.
    \end{medios}
    \begin{formasOrganizacion}
        %Se pone los que se necesiten
        \newItemFO{Clases Teóricas}{Desarrollo de los conceptos teóricos}
        \newItemFO{Seminarios}{Algo...}
        \newItemFO{Prácticas}{Algo...}
        \newItemFO{Laboratorio}{Aplicación de los conceptos vistos es clases teóricas.}
        \newItemFO{Otros}{Algo...}
    \end{formasOrganizacion}
    \begin{programacion}
        \newItemFO{Investigación Formativa}{Implementación de Sistema Computacional Web usando una base de datos relacional normalizada}
        \newItemFO{Responsabilidad Social}{Generar videos para la enseñanza de implementación de bases de datos y que sean disponibilizados de la población}
    \end{programacion}
    \begin{segumientoAprendizaje}
        Aquí va el seguimiento del aprendizaje
    \end{segumientoAprendizaje}
\end{estrategiasEnsenanza}


\begin{cronogramaAcademico}
    \newItemCA{Tema1}{Edward Hinojosa Cárdenas}{10}  %Tema/Evaluación - Docente - Porcentaje acumulado
    \newItemCA{Tema2}{Edward Hinojosa Cárdenas}{16}
    \newItemCA{Tema3}{Edward Hinojosa Cárdenas}{20}
    \newItemCA{Tema4}{Edward Hinojosa Cárdenas}{25}
    \newItemCA{Tema5}{Edward Hinojosa Cárdenas}{33}
    \newItemCA{Tema6}{Edward Hinojosa Cárdenas}{37}
    \newItemCA{Tema7}{Edward Hinojosa Cárdenas}{40}
    \newItemCA{Tema8}{Edward Hinojosa Cárdenas}{45}
    \newItemCA{Tema9}{Edward Hinojosa Cárdenas}{50}
    \newItemCA{Tema10}{Edward Hinojosa Cárdenas}{53}
    \newItemCA{Tema11}{Edward Hinojosa Cárdenas}{58}
    \newItemCA{Tema12}{Edward Hinojosa Cárdenas}{64}
    \newItemCA{Tema13}{Edward Hinojosa Cárdenas}{69}
    \newItemCA{Tema14}{Edward Hinojosa Cárdenas}{76}
    \newItemCA{Tema15}{Edward Hinojosa Cárdenas}{84}
    \newItemCA{Tema16}{Edward Hinojosa Cárdenas}{93}
    \newItemCA{Tema17}{Edward Hinojosa Cárdenas}{100}
\end{cronogramaAcademico}

\begin{estrategiasEvaluacion}
    \begin{evaluacionContinua}
        Práctica y Laboratorios en cada clase sobre los temas realizados, tanto para el primer parcial (EC1), segundo parcial (EC2) y tercer parcial (EC3).
    \end{evaluacionContinua}
    \begin{evaluacionPeriodica}
        \newItemEP{Primer Examen}{ponderación}
        \newItemEP{Segundo Examen}{ponderación}
        \newItemEP{Tercer Examen}{}
    \end{evaluacionPeriodica}
    \begin{cronogramaEvaluacion}
        \newItemCE{11/5/2019}{11/5/2019}{11/5/2019}{30\%}
        \newItemCE{11/8/2019}{11/8/2019}{11/8/2019}{30\%}
        \newItemCE{11/12/2019}{11/12/2019}{11/12/2019}{40\%}
    \end{cronogramaEvaluacion}
    \begin{tipoEvaluacion}
        Tipo de evaluación
    \end{tipoEvaluacion}
    \begin{instrumentosEvaluacion}
        Instrumentos de evaluación
    \end{instrumentosEvaluacion}
\end{estrategiasEvaluacion}

\begin{requisitosAprobacion}
\item El alumno tendrá derecho a observar o en su defecto a ratificar las notas consignadas en sus evaluaciones, después de ser entregadas las mismas por parte del profesor, salvo el vencimiento de plazos para culminación del semestre académico, luego del mismo, no se admitirán reclamaciones,
alumno que no se haga presente en el día establecido, perderá su derecho a reclamo.
\item Para aprobar ...
\end{requisitosAprobacion}

\bibliography{CS370.bib}
\bibliographystyle{apalike}

\fecha
\firma

\end{document}


