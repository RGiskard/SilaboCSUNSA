\documentclass[a4paper,8pt]{article}
\usepackage[utf8]{inputenc}
\usepackage{tabularx}
\usepackage{setspace}
\usepackage{../../silabus}

\begin{document}

%%Setear variables
\setPeriodoAcademico{2018-B}
\setNombreAsignatura{Bioinformática}
\setNombreProfesor{}
\setGradoProfesorAbreviado{}
\sylabusHeader

\academicaTable
{Ciencia de la Computación} %Escuela Profesional
{CB1705164} %Código de la asignatura
{9$^{no}$ Semestre.} %Número del semestre
{Semestral} %Características
{17 Semanas} %Duración
{2 HT} %Número de horas teóricas
{2 HP} %Número de horas prácticas
{0} %Número de horas seminarios
{2 HL}  %Número de horas laboratorio
{2} %Número de horas Teórico-práctico
{4} %Número de créditos
{MA1703241} % Prerrequisitos (separados por comas)

\administrativaTable
{Doctor} %Grado académico del profesor
{Ingeniería de Sistemas e Informática} %Departamento académico
{6} %Número de horas totales
{2} %Número de horas - lunes
{-} %Número de horas - martes
{2} %Número de horas - miercoles
{2} %Número de horas - jueves
{-} %Número de horas - viernes
{101} %Aula de clase - lunes
{-} %Aula de clase - martes
{101} %Aula de clase - miercoles
{101} %Aula de clase - jueves
{-} %Aula de clase - viernes


\begin{fundamentacion}
El uso de métodos computacionales en las ciencias biológicas se ha convertido en una de las herramientas claves para el campo de la biología molecular, siendo parte fundamental en las investigaciones de esta área. 
\\
En Biología Molecular, existen diversas aplicaciones que involucran tanto al ADN, al análisis de proteínas o al secuenciamiento del genoma humano, que dependen de métodos computacionales. Muchos de estos problemas son realmente complejos y tratan con grandes conjuntos de datos. 
\\
Este curso puede ser aprovechado para ver casos de uso concretos de varias áreas de conocimiento de Ciencia de la Computacion como: Lenguajes de Programación (PL), Algoritmos y Complejidad (AL), Probabilidades y Estadística, Manejo de Información (IM), Sistemas Inteligentes (IS).

\end{fundamentacion}

\begin{sumilla}
\item Introducción a la Biología Molecular
\item Comparación de Secuencias
\item Árboles Filogenéticos
\item Ensamblaje de Secuencias de ADN
\item Estructuras secundarias y terciarias
\item Modelos Probabilísticos en Biología Molecular

\end{sumilla}

\begin{competenciasAsignatura}
\item \ShowCompetence{C1}{a,b}
\item \ShowCompetence{C3}{b,l}
\item \ShowCompetence{C5}{a,b}

\end{competenciasAsignatura}

\begin{contenidos}


%inicio unidad
\nextUnidad{Introducción a la Biología Molecular}
\nextCapitulo{Introducción a la Biología Molecular}
\nextTema{Revisión de la química orgánica: moléculas y macromoléculas, azúcares, acidos nucleicos, nuclótidos, ARN, ADN, proteínas, aminoácidos y nivels de estructura en las proteinas.}
\nextTema{El dogma de la vida: del ADN a las proteinas, transcripción, traducción, síntesis de proteinas}
\nextTema{Estudio del genoma: Mapas y secuencias, técnicas específicas}

\cite{Clote2000}, \cite{Setubal1997} 


%inicio unidad
\nextUnidad{Comparación de Secuencias}
\nextCapitulo{Comparación de Secuencias}
\nextTema{Secuencias de nucléotidos y secuencias de aminoácidos.}
\nextTema{Alineamiento de secuencias, el problema de alineamiento por pares, búsqueda exhaustiva, Programación dinámica, alineamiento global, alineamiento local, penalización por gaps}
\nextTema{Comparación de múltiples secuencias: suma de pares, análisis de complejidad por programación dinámica, heurísticas de alineamiento, algoritmo estrella, algoritmos de alineamiento progresivo.}

\cite{Clote2000}, \cite{Setubal1997}, \cite{Pevzner2000} 


%inicio unidad
\nextUnidad{Árboles Filogenéticos}
\nextCapitulo{Árboles Filogenéticos}
\nextTema{Filogenia: Introducción y relaciones filogenéticas.}
\nextTema{Arboles Filogenéticos: definición, tipo de árboles, problema de búsqueda y reconstrucción de árboles}
\nextTema{Métodos de Reconstrucción: métodos por parsimonia, métodos por distancia, métodos por máxima verosimilitud, confianza de los árboles reconstruidos}

\cite{Clote2000}, \cite{Setubal1997}, \cite{Pevzner2000} 


%inicio unidad
\nextUnidad{Ensamblaje de Secuencias de ADN}
\nextCapitulo{Ensamblaje de Secuencias de ADN}
\nextTema{Fundamento biológico: caso ideal, dificultades, métodos alternativos para secuenciamiento de ADN}
\nextTema{Modelos formales de ensamblaje: \textit{Shortest Common Superstring}, \textit{Reconstruction}, \textit{Multicontig}}
\nextTema{Algoritmos para ensamblaje de secuencias: representación de overlaps, caminos para crear \textit{superstrings}, algoritmo voraz, grafos acíclicos.}
\nextTema{Heurísticas para ensamblaje: búsqueda de sobreposiciones, ordenación de fragmentos, alineamientos y consenso.}

\cite{Setubal1997}, \cite{Aluru2006} 


%inicio unidad
\nextUnidad{Estructuras secundarias y terciarias}
\nextCapitulo{Estructuras secundarias y terciarias}
\nextTema{Estructuras moleculares: primaria, secundaria, terciaria, cuaternaria.}
\nextTema{Predicción de estructuras secundarias de ARN: modelo formal, energia de pares, estructuras con bases independientes, solución con Programación Dinámica, estructuras con bucles.}
\nextTema{{\it Protein folding}: Estructuras en proteinas, problema de \textit{protein folding}.}
\nextTema{{\it Protein Threading}: Definiciones, Algoritmo \textit{Branch \& Bound}, \textit{Branch \& Bound} para \textit{protein threading}.}
\nextTema{{\it Structural Alignment}: definiciones, algoritmo DALI}

\cite{Setubal1997}, \cite{Clote2000}, \cite{Aluru2006} 


%inicio unidad
\nextUnidad{Modelos Probabilísticos en Biología Molecular}
\nextCapitulo{Modelos Probabilísticos en Biología Molecular}
\nextTema{Probabilidad: Variables aleatorias, Cadenas de Markov, Algoritmo de Metropoli-Hasting, Campos Aleatorios de Markov y Muestreador de Gibbs, Máxima Verosimilitud.}
\nextTema{Modelos Ocultos de Markov (HMM), estimación de parámetros, algoritmo de Viterbi y método Baul-Welch, Aplicación en alineamientos de pares y múltiples, en detección de Motifs en proteínas, en ADN eucariótico, en familias de secuencias.}
\nextTema{Filogenia Probabilística: Modelos probabilísticos de evolución, verosimilitud de alineamientos, verosimilitud para inferencia, comparación de métodos probailísticos y no probabilísticos}

\cite{Durbin1998}, \cite{Clote2000}, \cite{Aluru2006}, \cite{Krogh1994} 





\end{contenidos}




\begin{estrategiasEnsenanza}
    \begin{metodos}
        Método expositivo en las clases teóricas \\
        Método de elaboración conjunta en los seminarios taller y en la elaboración del proyecto de investigación.
    \end{metodos}
    \begin{medios}
        Pizarra acrílica, plumones, cañón multimedia, material de laboratorio, videos, software.
    \end{medios}
    \begin{formasOrganizacion}
        %Se pone los que se necesiten
        \newItemFO{Clases Teóricas}{Desarrollo de los conceptos teóricos}
        \newItemFO{Seminarios}{Algo...}
        \newItemFO{Prácticas}{Algo...}
        \newItemFO{Laboratorio}{Aplicación de los conceptos vistos es clases teóricas.}
        \newItemFO{Otros}{Algo...}
    \end{formasOrganizacion}
    \begin{programacion}
        \newItemFO{Investigación Formativa}{Implementación de Sistema Computacional Web usando una base de datos relacional normalizada}
        \newItemFO{Responsabilidad Social}{Generar videos para la enseñanza de implementación de bases de datos y que sean disponibilizados de la población}
    \end{programacion}
    \begin{segumientoAprendizaje}
        Aquí va el seguimiento del aprendizaje
    \end{segumientoAprendizaje}
\end{estrategiasEnsenanza}


\begin{cronogramaAcademico}
    \newItemCA{Tema1}{Edward Hinojosa Cárdenas}{10}  %Tema/Evaluación - Docente - Porcentaje acumulado
    \newItemCA{Tema2}{Edward Hinojosa Cárdenas}{16}
    \newItemCA{Tema3}{Edward Hinojosa Cárdenas}{20}
    \newItemCA{Tema4}{Edward Hinojosa Cárdenas}{25}
    \newItemCA{Tema5}{Edward Hinojosa Cárdenas}{33}
    \newItemCA{Tema6}{Edward Hinojosa Cárdenas}{37}
    \newItemCA{Tema7}{Edward Hinojosa Cárdenas}{40}
    \newItemCA{Tema8}{Edward Hinojosa Cárdenas}{45}
    \newItemCA{Tema9}{Edward Hinojosa Cárdenas}{50}
    \newItemCA{Tema10}{Edward Hinojosa Cárdenas}{53}
    \newItemCA{Tema11}{Edward Hinojosa Cárdenas}{58}
    \newItemCA{Tema12}{Edward Hinojosa Cárdenas}{64}
    \newItemCA{Tema13}{Edward Hinojosa Cárdenas}{69}
    \newItemCA{Tema14}{Edward Hinojosa Cárdenas}{76}
    \newItemCA{Tema15}{Edward Hinojosa Cárdenas}{84}
    \newItemCA{Tema16}{Edward Hinojosa Cárdenas}{93}
    \newItemCA{Tema17}{Edward Hinojosa Cárdenas}{100}
\end{cronogramaAcademico}

\begin{estrategiasEvaluacion}
    \begin{evaluacionContinua}
        Práctica y Laboratorios en cada clase sobre los temas realizados, tanto para el primer parcial (EC1), segundo parcial (EC2) y tercer parcial (EC3).
    \end{evaluacionContinua}
    \begin{evaluacionPeriodica}
        \newItemEP{Primer Examen}{ponderación}
        \newItemEP{Segundo Examen}{ponderación}
        \newItemEP{Tercer Examen}{}
    \end{evaluacionPeriodica}
    \begin{cronogramaEvaluacion}
        \newItemCE{11/5/2019}{11/5/2019}{11/5/2019}{30\%}
        \newItemCE{11/8/2019}{11/8/2019}{11/8/2019}{30\%}
        \newItemCE{11/12/2019}{11/12/2019}{11/12/2019}{40\%}
    \end{cronogramaEvaluacion}
    \begin{tipoEvaluacion}
        Tipo de evaluación
    \end{tipoEvaluacion}
    \begin{instrumentosEvaluacion}
        Instrumentos de evaluación
    \end{instrumentosEvaluacion}
\end{estrategiasEvaluacion}

\begin{requisitosAprobacion}
\item El alumno tendrá derecho a observar o en su defecto a ratificar las notas consignadas en sus evaluaciones, después de ser entregadas las mismas por parte del profesor, salvo el vencimiento de plazos para culminación del semestre académico, luego del mismo, no se admitirán reclamaciones,
alumno que no se haga presente en el día establecido, perderá su derecho a reclamo.
\item Para aprobar ...
\end{requisitosAprobacion}

\bibliography{CB309.bib}
\bibliographystyle{apalike}

\fecha
\firma

\end{document}


