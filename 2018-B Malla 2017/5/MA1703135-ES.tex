\documentclass[a4paper,8pt]{article}
\usepackage[utf8]{inputenc}
\usepackage{tabularx}
\usepackage{setspace}
\usepackage{../../silabus}

\begin{document}

%%Setear variables
\setPeriodoAcademico{2018-B}
\setNombreAsignatura{Ecuaciones Diferenciales}
\setNombreProfesor{}
\setGradoProfesorAbreviado{}
\sylabusHeader

\academicaTable
{Ciencia de la Computación} %Escuela Profesional
{MA1703135} %Código de la asignatura
{5$^{to}$ Semestre.} %Número del semestre
{Semestral} %Características
{17 Semanas} %Duración
{2 HT} %Número de horas teóricas
{2 HP} %Número de horas prácticas
{0} %Número de horas seminarios
{2 HL}  %Número de horas laboratorio
{2} %Número de horas Teórico-práctico
{4} %Número de créditos
{MA1702227,MA1702121} % Prerrequisitos (separados por comas)

\administrativaTable
{Doctor} %Grado académico del profesor
{Ingeniería de Sistemas e Informática} %Departamento académico
{6} %Número de horas totales
{2} %Número de horas - lunes
{-} %Número de horas - martes
{2} %Número de horas - miercoles
{2} %Número de horas - jueves
{-} %Número de horas - viernes
{101} %Aula de clase - lunes
{-} %Aula de clase - martes
{101} %Aula de clase - miercoles
{101} %Aula de clase - jueves
{-} %Aula de clase - viernes


\begin{fundamentacion}
El curso de Cálculo Integral contribuirá a la adquisición de la competencia Aprender a aprender, en relación con el perfil de egresado de Estudios Generales Ciencias y con el objetivo (a) de ABET ``Capacidad de aplicar conocimientos de matemáticas, ciencias e ingeniería'' en la medida en que al finalizar la asignatura el estudiante será capaz de:

\end{fundamentacion}

\begin{sumilla}
\item Capítulo 1: La integral definida (10 horas)
\item Capítulo 2: Métodos de integración (10 horas)
\item Capítulo 3: Aplicaciones de la integral (22 horas)
\item Capítulo 4: Ecuaciones diferenciales (8 horas)
\item Capítulo 5: Integrales impropias

\end{sumilla}

\begin{competenciasAsignatura}
\item \ShowCompetence{C20}{d,h}

\end{competenciasAsignatura}

\begin{contenidos}


%inicio unidad
\nextUnidad{Capítulo 1: La integral definida (10 horas)}
\nextCapitulo{Capítulo 1: La integral definida (10 horas)}
\nextTema{Notación Sigma (Sumas).}
\nextTema{Fórmulas de algunas sumas especiales.}
\nextTema{Idea intuitiva del cálculo del área de una región plana limitada por la gráfica de una función no negativa en un intervalo acotado.}
\nextTema{Partición regular.}
\nextTema{Sumas de Riemann.}
\nextTema{La integral definida como límite de una suma de Riemann.}
\nextTema{Propiedades de la integral definida.}
\nextTema{Definiciones de función acotada y continua por tramos (seccionalmente continua).}
\nextTema{El teorema del valor medio.}
\nextTema{Interpretación geométrica y demostración.}
\nextTema{Definición de valor promedio de una función en un intervalo cerrado.}
\nextTema{Teoremas fundamentales del cálculo.}

 


%inicio unidad
\nextUnidad{Capítulo 2: Métodos de integración (10 horas)}
\nextCapitulo{Capítulo 2: Métodos de integración (10 horas)}
\nextTema{Integración por partes para integrales definidas.}
\nextTema{Demostración de integrales definidas por recurrencia.}
\nextTema{Integrales trigonométricas: potencias de seno por coseno, tangente por secante y productos de seno y coseno con ángulos diferentes.}
\nextTema{Integración por sustitución trigonométrica.}
\nextTema{Aplicación a integrales con potencias fraccionarias.}
\nextTema{Integración de funciones racionales con el método de fracciones parciales}

 


%inicio unidad
\nextUnidad{Capítulo 3: Aplicaciones de la integral (22 horas)}
\nextCapitulo{Capítulo 3: Aplicaciones de la integral (22 horas)}
\nextTema{Área entre dos curvas en coordenadas cartesianas.}
\nextTema{Cálculo de volúmenes de sólidos mediante secciones transversales.}
\nextTema{Cálculo de volúmenes de sólidos de revolución: método del disco y del anillo.}
\nextTema{Método de las cortezas cilíndricas.}
\nextTema{Longitud de arco de una curva.}
\nextTema{Área de la superficie de revolución generada por una curva.}
\nextTema{Momentos y centro de masa.}
\nextTema{Centroide de una región plana.}
\nextTema{Teorema de Pappus para hallar el volumen de un sólido de revolución.}
\nextTema{Coordenadas polares.}
\nextTema{Relaciones entre coordenadas cartesianas y polares.}
\nextTema{Ecuaciones polares y ecuaciones equivalentes.}
\nextTema{Gráfica de ecuaciones en coordenadas polares.}
\nextTema{Área de regiones plana y longitud de arco en coordenadas polares.}
\nextTema{La fórmula de Taylor con resto de Lagrange.}
\nextTema{Aproximación de cfunciones y Cálculo aproximado de integrales mediante polinomios de 				Taylor.}

 


%inicio unidad
\nextUnidad{Capítulo 4: Ecuaciones diferenciales (8 horas)}
\nextCapitulo{Capítulo 4: Ecuaciones diferenciales (8 horas)}
\nextTema{Definición de EDO lineal de orden n.}
\nextTema{Caso particular: EDO lineal de orden 2.}
\nextTema{Solución de ecuaciones lineales no homogéneas lineales con coeficientes constantes.}
\nextTema{Teorema de existencia y unicidad para EDOs de orden $n$.}
\nextTema{El método de reducción de orden. Independencia lineal y wronskiano.}
\nextTema{Obtención de una solución particular de la ecuación no homogénea.}
\nextTema{El método de los coeficientes indeterminados y el de variación de parámetros.}
\nextTema{Aplicaciones: Movimientos vibratorios de sistemas mecánicos. Movimiento armónico simple.}
\nextTema{Movimiento vibratorio con amortiguamiento.}
\nextTema{El fenómeno de resonancia. Problemas de circuitos eléctricos y el péndulo simple.}

 


%inicio unidad
\nextUnidad{Capítulo 5: Integrales impropias}
\nextCapitulo{Capítulo 5: Integrales impropias}
\nextTema{Definición de integrales impropias (en intervalos no acotados y con asíntotas verticales).}
\nextTema{Criterios de convergencia: comparación, paso al límite y convergencia absoluta.}
\nextTema{Ejemplos de convergencia de integrales impropias que dependan de un parámetro}

 





\end{contenidos}




\begin{estrategiasEnsenanza}
    \begin{metodos}
        Método expositivo en las clases teóricas \\
        Método de elaboración conjunta en los seminarios taller y en la elaboración del proyecto de investigación.
    \end{metodos}
    \begin{medios}
        Pizarra acrílica, plumones, cañón multimedia, material de laboratorio, videos, software.
    \end{medios}
    \begin{formasOrganizacion}
        %Se pone los que se necesiten
        \newItemFO{Clases Teóricas}{Desarrollo de los conceptos teóricos}
        \newItemFO{Seminarios}{Algo...}
        \newItemFO{Prácticas}{Algo...}
        \newItemFO{Laboratorio}{Aplicación de los conceptos vistos es clases teóricas.}
        \newItemFO{Otros}{Algo...}
    \end{formasOrganizacion}
    \begin{programacion}
        \newItemFO{Investigación Formativa}{Implementación de Sistema Computacional Web usando una base de datos relacional normalizada}
        \newItemFO{Responsabilidad Social}{Generar videos para la enseñanza de implementación de bases de datos y que sean disponibilizados de la población}
    \end{programacion}
    \begin{segumientoAprendizaje}
        Aquí va el seguimiento del aprendizaje
    \end{segumientoAprendizaje}
\end{estrategiasEnsenanza}


\begin{cronogramaAcademico}
    \newItemCA{Tema1}{Edward Hinojosa Cárdenas}{10}  %Tema/Evaluación - Docente - Porcentaje acumulado
    \newItemCA{Tema2}{Edward Hinojosa Cárdenas}{16}
    \newItemCA{Tema3}{Edward Hinojosa Cárdenas}{20}
    \newItemCA{Tema4}{Edward Hinojosa Cárdenas}{25}
    \newItemCA{Tema5}{Edward Hinojosa Cárdenas}{33}
    \newItemCA{Tema6}{Edward Hinojosa Cárdenas}{37}
    \newItemCA{Tema7}{Edward Hinojosa Cárdenas}{40}
    \newItemCA{Tema8}{Edward Hinojosa Cárdenas}{45}
    \newItemCA{Tema9}{Edward Hinojosa Cárdenas}{50}
    \newItemCA{Tema10}{Edward Hinojosa Cárdenas}{53}
    \newItemCA{Tema11}{Edward Hinojosa Cárdenas}{58}
    \newItemCA{Tema12}{Edward Hinojosa Cárdenas}{64}
    \newItemCA{Tema13}{Edward Hinojosa Cárdenas}{69}
    \newItemCA{Tema14}{Edward Hinojosa Cárdenas}{76}
    \newItemCA{Tema15}{Edward Hinojosa Cárdenas}{84}
    \newItemCA{Tema16}{Edward Hinojosa Cárdenas}{93}
    \newItemCA{Tema17}{Edward Hinojosa Cárdenas}{100}
\end{cronogramaAcademico}

\begin{estrategiasEvaluacion}
    \begin{evaluacionContinua}
        Práctica y Laboratorios en cada clase sobre los temas realizados, tanto para el primer parcial (EC1), segundo parcial (EC2) y tercer parcial (EC3).
    \end{evaluacionContinua}
    \begin{evaluacionPeriodica}
        \newItemEP{Primer Examen}{ponderación}
        \newItemEP{Segundo Examen}{ponderación}
        \newItemEP{Tercer Examen}{}
    \end{evaluacionPeriodica}
    \begin{cronogramaEvaluacion}
        \newItemCE{11/5/2019}{11/5/2019}{11/5/2019}{30\%}
        \newItemCE{11/8/2019}{11/8/2019}{11/8/2019}{30\%}
        \newItemCE{11/12/2019}{11/12/2019}{11/12/2019}{40\%}
    \end{cronogramaEvaluacion}
    \begin{tipoEvaluacion}
        Tipo de evaluación
    \end{tipoEvaluacion}
    \begin{instrumentosEvaluacion}
        Instrumentos de evaluación
    \end{instrumentosEvaluacion}
\end{estrategiasEvaluacion}

\begin{requisitosAprobacion}
\item El alumno tendrá derecho a observar o en su defecto a ratificar las notas consignadas en sus evaluaciones, después de ser entregadas las mismas por parte del profesor, salvo el vencimiento de plazos para culminación del semestre académico, luego del mismo, no se admitirán reclamaciones,
alumno que no se haga presente en el día establecido, perderá su derecho a reclamo.
\item Para aprobar ...
\end{requisitosAprobacion}

\bibliography{MA312.bib}
\bibliographystyle{apalike}

\fecha
\firma

\end{document}


