\documentclass[12pt]{article}
\usepackage[utf8]{inputenc}
\usepackage{vmargin}
\usepackage{fancyhdr}
\usepackage{graphics}
\usepackage{../../icacit}
\usepackage[margin=2.5cm,headheight=200pt,includeheadfoot]{geometry}

\setpapersize{A4}
\setmargins{2.5cm}       % margen izquierdo
{0.5cm}                        % margen superior
{16.5cm}                      % anchura del texto
{23.42cm}                    % altura del texto
{10pt}                           % altura de los encabezados
{1cm}                           % espacio entre el texto y los encabezados
{0pt}                             % altura del pie de página
{2cm}                           % espacio entre el texto y el pie de página


\begin{document}

\sylabusHeader
\sylabusTitle

\curso
{1004247} %Código del curso
{COMPILADORES} %Nombre del curso
{2018-B} %Semestre

\creditosHoras
{4} %Número de creditos
{2} %Horas teoría
{2} %Horas práctica
{2} %Horas teórico-practica
{} %Horas laboratorio
{6} %Total de horas

\instructor
{Dr. Yuber Elmer Velazco Paredes}

\libro
{Construccion de Compiladores Principios y Practica} %Título del libro
{Kenneth C. Louden} %Autor del libro
{2004} %Año del libro

\libroSecundario
{Compiladores. Principios, técnicas y herramientas} %Título del libro
{Alfred Aho and Mónica Lam and Ravi Sethi and Jeffrey D. Ullman} %Autor del libro
{2008} %Año del libro

\begin{datosCurso}
    \begin{descripcion}
        El curso de Compiladores tiene como propósito que el estudiante aprenda los principios fundamentales de análisis y síntesis de compiladores y/o interpretes en la solución de problemas computacionales.
        
    \end{descripcion}
    \begin{requisitos}
        \newItemReq{1004141}{Lenguajes de Programación}
        
    \end{requisitos}
    \ObligatorioElectivo
    {X} %Marcar esta si es obligatorio 
    {} %Marcar esta si es electivo
\end{datosCurso}

\begin{objetivosCurso}
    \begin{resultadosEspecificos}
        El  curso  está orientado a que el estudiante desarrolle un compilador y/o intérprete basado en el diseño de una gramática y utilización de alguna herramientas que facilite su elaboración.
    \end{resultadosEspecificos}
    \begin{resultadosEstudiante}
        i) La capacidad de aplicar fundamentos matemáticos, principios algorítmicos y teoría de ciencias de la computación en el modelamiento y diseño de sistemas basados en computadora de modo que demuestren la comprensión de las ventajas y desventajas involucradas en las opciones de diseño (1:Comprende)
    \end{resultadosEstudiante}
\end{objetivosCurso}

\begin{ListaTemas}
\nextUnidad{Visión general de los compiladores}
\nextUnidad{Análisis Léxico}
\nextUnidad{Análisis Sintáctico}
\nextUnidad{Análisis Semántico}
\nextUnidad{Generación de código}
\nextUnidad{Manejo de errores}
\end{ListaTemas}


\end{document}