\documentclass[a4paper,8pt]{article}
\usepackage[utf8]{inputenc}
\usepackage{tabularx}
\usepackage{setspace}
\usepackage{../../silabus}

\begin{document}

%%Setear variables
\setPeriodoAcademico{2018-B}
\setNombreAsignatura{Relaciones Humanas}
\setNombreProfesor{}
\setGradoProfesorAbreviado{}
\sylabusHeader

\academicaTable
{Ciencia de la Computación} %Escuela Profesional
{FG1704253} %Código de la asignatura
{8$^{vo}$ Semestre.} %Número del semestre
{Semestral} %Características
{17 Semanas} %Duración
{1 HT} %Número de horas teóricas
{2 HP} %Número de horas prácticas
{0} %Número de horas seminarios
{}  %Número de horas laboratorio
{2} %Número de horas Teórico-práctico
{2} %Número de créditos
{FG1702229} % Prerrequisitos (separados por comas)

\administrativaTable
{Doctor} %Grado académico del profesor
{Ingeniería de Sistemas e Informática} %Departamento académico
{6} %Número de horas totales
{2} %Número de horas - lunes
{-} %Número de horas - martes
{2} %Número de horas - miercoles
{2} %Número de horas - jueves
{-} %Número de horas - viernes
{101} %Aula de clase - lunes
{-} %Aula de clase - martes
{101} %Aula de clase - miercoles
{101} %Aula de clase - jueves
{-} %Aula de clase - viernes


\begin{fundamentacion}
El estudio de la Filosofía en la universidad, se presenta como un espacio de reflexión constante sobre el ser y el quehacer del ser humano en el mundo. Así mismo, proporciona las herramientas académicas necesarias para la adquisición del pensamiento formal y la actitud crítica frente a las corrientes relativistas que nos alejan de la Verdad.
La formación filosófica aporta considerablemente al cultivo de los saberes, capacidades y potencialidades humanas, de tal manera que facilita al ser humano encontrar el camino hacia la Verdad plena.

\end{fundamentacion}

\begin{sumilla}
\item 
\item 
\item 
\item 
\item 

\end{sumilla}

\begin{competenciasAsignatura}
\item \ShowCompetence{C20}{n}
\item \ShowCompetence{C21}{e,n}

\end{competenciasAsignatura}

\begin{contenidos}


%inicio unidad
\nextUnidad{}
\nextCapitulo{}
\nextTema{Importancia de la Filosofia}
\nextTema{Filosofía: definición etimológica y real.}
\nextTema{El asombro como comienzo del filosofar.}
\nextTema{El ocio como condición para la filosofía.}
\nextTema{La filosofía como sabiduría natural.}
\nextTema{Condiciones morales del filosofar.}
\nextTema{Filosofía y otros conocimientos.}
\nextTema{Aproximación histórica: Antigua, media, moderna y contemporánea.}

\cite{Pieper} 


%inicio unidad
\nextUnidad{}
\nextCapitulo{}
\nextTema{Características generales de la Antropología filosófica. Definiciones, objetos, métodos y relación con otros saberes.}
\nextTema{Visiones reduccionistas: materialismo y espiritualismo.}
\nextTema{Visión integral del ser humano.}
\nextTema{La persona humana. Definición. Unidad sustancial del cuerpo y el espíritu.}

\cite{Amerio}, \cite{Acodesi} 


%inicio unidad
\nextUnidad{}
\nextCapitulo{}
\nextTema{Características generales del conocimiento humano.}
\nextTema{Discusión con otras posturas: el escepticismo y el relativismo; racionalismo y empirismo.}
\nextTema{La verdad: lógica y ontológica.}

\cite{Zanotti}, \cite{Platon}, \cite{Pieper} 


%inicio unidad
\nextUnidad{}
\nextCapitulo{}
\nextTema{Características generales: etimología, moral y ética, objeto, tipo de conocimiento.}
\nextTema{Criterios de moralidad.}
\nextTema{Fuentes de la moralidad.}
\nextTema{Relativismo ético.}
\nextTema{El bien. El fin último. La felicidad.}
\nextTema{Virtudes}

\cite{AristotelesE} 


%inicio unidad
\nextUnidad{}
\nextCapitulo{}
\nextTema{La metafísica como estudio del ser.}
\nextTema{Los trascendentales}
\nextTema{La estructura del ente finito.Sustancia-Accidente, Materia-Forma, Acto-Potencia, Esencia-Acto de ser.}
\nextTema{La causalidad. La existencia de Dios. La creación y sus implicancias.}

\cite{Gomez}, \cite{Alvira} 





\end{contenidos}




\begin{estrategiasEnsenanza}
    \begin{metodos}
        Método expositivo en las clases teóricas \\
        Método de elaboración conjunta en los seminarios taller y en la elaboración del proyecto de investigación.
    \end{metodos}
    \begin{medios}
        Pizarra acrílica, plumones, cañón multimedia, material de laboratorio, videos, software.
    \end{medios}
    \begin{formasOrganizacion}
        %Se pone los que se necesiten
        \newItemFO{Clases Teóricas}{Desarrollo de los conceptos teóricos}
        \newItemFO{Seminarios}{Algo...}
        \newItemFO{Prácticas}{Algo...}
        \newItemFO{Laboratorio}{Aplicación de los conceptos vistos es clases teóricas.}
        \newItemFO{Otros}{Algo...}
    \end{formasOrganizacion}
    \begin{programacion}
        \newItemFO{Investigación Formativa}{Implementación de Sistema Computacional Web usando una base de datos relacional normalizada}
        \newItemFO{Responsabilidad Social}{Generar videos para la enseñanza de implementación de bases de datos y que sean disponibilizados de la población}
    \end{programacion}
    \begin{segumientoAprendizaje}
        Aquí va el seguimiento del aprendizaje
    \end{segumientoAprendizaje}
\end{estrategiasEnsenanza}


\begin{cronogramaAcademico}
    \newItemCA{Tema1}{Edward Hinojosa Cárdenas}{10}  %Tema/Evaluación - Docente - Porcentaje acumulado
    \newItemCA{Tema2}{Edward Hinojosa Cárdenas}{16}
    \newItemCA{Tema3}{Edward Hinojosa Cárdenas}{20}
    \newItemCA{Tema4}{Edward Hinojosa Cárdenas}{25}
    \newItemCA{Tema5}{Edward Hinojosa Cárdenas}{33}
    \newItemCA{Tema6}{Edward Hinojosa Cárdenas}{37}
    \newItemCA{Tema7}{Edward Hinojosa Cárdenas}{40}
    \newItemCA{Tema8}{Edward Hinojosa Cárdenas}{45}
    \newItemCA{Tema9}{Edward Hinojosa Cárdenas}{50}
    \newItemCA{Tema10}{Edward Hinojosa Cárdenas}{53}
    \newItemCA{Tema11}{Edward Hinojosa Cárdenas}{58}
    \newItemCA{Tema12}{Edward Hinojosa Cárdenas}{64}
    \newItemCA{Tema13}{Edward Hinojosa Cárdenas}{69}
    \newItemCA{Tema14}{Edward Hinojosa Cárdenas}{76}
    \newItemCA{Tema15}{Edward Hinojosa Cárdenas}{84}
    \newItemCA{Tema16}{Edward Hinojosa Cárdenas}{93}
    \newItemCA{Tema17}{Edward Hinojosa Cárdenas}{100}
\end{cronogramaAcademico}

\begin{estrategiasEvaluacion}
    \begin{evaluacionContinua}
        Práctica y Laboratorios en cada clase sobre los temas realizados, tanto para el primer parcial (EC1), segundo parcial (EC2) y tercer parcial (EC3).
    \end{evaluacionContinua}
    \begin{evaluacionPeriodica}
        \newItemEP{Primer Examen}{ponderación}
        \newItemEP{Segundo Examen}{ponderación}
        \newItemEP{Tercer Examen}{}
    \end{evaluacionPeriodica}
    \begin{cronogramaEvaluacion}
        \newItemCE{11/5/2019}{11/5/2019}{11/5/2019}{30\%}
        \newItemCE{11/8/2019}{11/8/2019}{11/8/2019}{30\%}
        \newItemCE{11/12/2019}{11/12/2019}{11/12/2019}{40\%}
    \end{cronogramaEvaluacion}
    \begin{tipoEvaluacion}
        Tipo de evaluación
    \end{tipoEvaluacion}
    \begin{instrumentosEvaluacion}
        Instrumentos de evaluación
    \end{instrumentosEvaluacion}
\end{estrategiasEvaluacion}

\begin{requisitosAprobacion}
\item El alumno tendrá derecho a observar o en su defecto a ratificar las notas consignadas en sus evaluaciones, después de ser entregadas las mismas por parte del profesor, salvo el vencimiento de plazos para culminación del semestre académico, luego del mismo, no se admitirán reclamaciones,
alumno que no se haga presente en el día establecido, perderá su derecho a reclamo.
\item Para aprobar ...
\end{requisitosAprobacion}

\bibliography{FG101.bib}
\bibliographystyle{apalike}

\fecha
\firma

\end{document}


